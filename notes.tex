\documentclass{amsart}

\usepackage{notes}

\title{BIRS-CMO Workshop on Equivariant TQFT}
\author{Notes by John S.\ Nolan, speakers listed below}

\begin{document}

\maketitle

\tableofcontents

\section{Karl Dovermann -- Equivariant real algebraic models}

In general, we want to ask: given a manifold, how much additional structure can we add?
How unique is this?
For example:

\begin{itemize}
  \item Triangulations may not exist and need not be unique (the ``Hauptvermutung'' is false).
  \item Smooth structures may not exist and need not be unique.
\end{itemize}

\subsection{Manifolds and algebraic geometry}

We are interested in understanding when a manifold can be equipped with the structure of a real algebraic variety.
Existence of such structures can be understood in terms of bordism.
These structures, when they exist, are not necessarily unique.

\begin{ex}
  The 2-torus $T^2$ has two inequivalent real algebraic structures:
  \begin{itemize}
    \item The standard picture $T^2 = (S^1)^2 \subset \RR^4$.
    \item We may also write $T^2 = \bset{(x, y, z) \in \RR^3}{x^2 + y^2 = x^2 + y^2 + z^2 + 3}$.
  \end{itemize}
\end{ex}

In fact, it's more interesting to consider the following question.
Let $M^m$ be a smooth manifold of dimension $m$, and let $\chi_T: M \to \Gr(RR^n, m)$ be a map classifying the tangent bundle of $M$.
Can we obtain a commutative diagram
\[
  \begin{tikzcd}
    M \dar["\sim"] \rar["\chi_T"] & \Gr(RR^n, m) \\
    X \ar[ur]
  \end{tikzcd}
\]
where $X$ is a real algebraic variety and $X \to \Gr(RR^n, m)$ is an entire rational map?
The answer may again be understood via bordism: we get a class $[M, \chi_T] \in \Nc_*(\Gr(RR^n, m))$.

\subsection{An equivariant problem}

What happens if we incorporate a compact Lie group $G$ into the story?
Let $\Xi$ be a real $G$-representation; then we want our variety $X$ to be a $G$-invariant subvariety of $X$.
We let $\Gr(\Xi, d) = \bset{L \in \Mat_{m \times m}}{L^2 = L, L^T = L, \tr L = d}$.

Let $M$ be a closed smooth $G$-manifold.
A (real) \emph{algebraic model} $X$ of $M$ is an algebraic $G$-variety equivariantly diffeomorphic to $M$.
Say this is \emph{strong} if all vector bundles are classified by equivariant rational maps.

\begin{qn}
  Do such exist?
\end{qn}

\begin{thm}
  As in the nonequivariant case, if a manifold is equivariantly cobordant to a non-singular real algebraic variety, then it is equivariantly diffeomorphic to a real algebraic variety.
\end{thm}

We end up looking at $\Nc_*^G(\prod^G \Gr(\Xi, d))$.
Existence of real algebraic models holds in the following cases:
\begin{itemize}
  \item True whenever $G$ is of odd order.
  \item True if $G = \ZZ_2 \times \dots \times \ZZ_2$ is an elementary abelian group.
  \item True if $G = \ZZ_{2^k}$.
\end{itemize}

\subsection{Working with 2-groups}

Let $G$ be a finite group and $G_2$ a Sylow 2-subgroup.
If all isotropy groups of $G \curvearrowright M$ are 2-groups, then Stong's theorem implies the following:
if $M$ admits a strong algebraic model as $G_2$-manifolds, then $M$ admits a strong algebraic model as $G$-manifolds.

To reduce to the case where all isotropy groups are 2-groups, we can blow up along subgroups.
This can be understood in differential geometry as removing the normal bundle to a subvariety and gluing in a projective bundle.

\begin{thm}
  If $A \subset M$ is $G$-invariant and $A$ and $\Bl_A M$ have strong algebraic models, then so does $M$.
\end{thm}

We can use this to get strong algebraic models in some key cases, e.g.\ if $G$ acts trivially on $M$.
More generally, this works if $M$ has only one isotropy type $H$ and every representation of $H$ is a restriction of representations of $G$.

These methods can also be used to obtain real algebraic models to the following cases:
\begin{itemize}
  \item True if $G_2$ is cyclic.
  \item True if $G$ is simple and $G_2$ is elementary abelian.
\end{itemize}

We'd still like to account for more $2$-groups, e.g.\ $Q_8$, $D_8$, and $\ZZ_4 \times \ZZ_4$.

\begin{ex}
  Let $G = \ZZ_4 \times \ZZ_4$, and suppose $M$ has only one isotropy type $H$.
  Let's first focus on the case $H = \ZZ_2 \times \ZZ_2$.
  We look at $\Nc^G_*[H](\prod \Gr(\Xi, d))$ and want to understand which classes are represented by algebraic varieties.
  One may pass to $\Nc^{G/H}_*[\textrm{free}](\prod \Gr(\Xi, d_i)^H)$ and then reduce to the non-equivariant case.
  Indeed, letting $\Fc$ be a component of $\prod \Gr(\Xi, d_i)^H$ and $W = G/H$, there is an isomorphism $\Nc_*^W[\textrm{free}](\Fc) \xrightarrow{\sim} \Nc_*(EW \times_W \Fc)$.
  The Thom construction gives a map $\Nc_*(EW \times_W \Fc) \to H_*(EW \times_W \Fc, \ZZ_2)$.
  These last homology groups may be computed via spectral sequences.
\end{ex}

There are still many open problems about understanding what happens for other 2-groups.

\section{Daniel L\'opez Neumann -- Quantum Invariants, Geometric Topology, and Non-Semisimplicity}

This is based on joint work with Roland van der Veen.

Recall that a knot $K$ is an embedding $S^1 \hookrightarrow S^3$.

We want to understand quantum invariants of knots (after Jones, Witten, Reshetikhin--Turaev, Drinfeld, etc.).
Let $\gfr$ be a simple Lie algebra and $V$ a $\gfr$-representation.
We may construct an isotropy invariant $I_V(K) \in \ZZ[q^{\pm 1}]$.
There are more general constructions which will be discussed tomorrow.

\begin{qn}
  What is the topological content of these quantum invariants?
\end{qn}

To understand this, we can think of the Alexander polynomial as a non-semisimple quantum invariant.
Thus we should try to understand non-semisimple quantum invariants.

\subsection{The Alexander polynomial}

Let $K \subset S^3$ be a knot, and let $X_K = S^3 \setminus K$.
The first homology $H_1(X_K)$ is just $\ZZ$, so this is not interesting.
However, we may write $H_1(X_K)$ as the abelianization of $\pi_1(X_K)$, which is more interesting.
Let $\hat{X_K}$ be the covering space of $X_K$ corresponding to the commutator subgroup $[\pi_1, \pi_1] \subset \pi_1(X_K)$.
In particular, the fibers of $\hat{X_K} \to X_K$ are $\ZZ$, so we get a $\ZZ[\ZZ] = \ZZ[t^{\pm 1}]$-action on $C_*(\hat{X_K})$.

We'll focus on the action $\ZZ[t^{\pm 1}] \curvearrowright H_1(\hat{X_K})$.
Write $H_1(\hat{X_K}) = \oplus_i \ZZ[t^{\pm 1}] / p_i(t)$, and let
\[
  \Delta_K(t) = \prod_i p_i(t) = \ord_{\ZZ[t^{\pm 1}]} H_1(\hat{X_K}).
\]
We may view the Alexander polynomial $\Delta_K$ as an invariant of 3-manifolds with $\ZZ$-action.
Alternatively, $\Delta_K$ generalizes to an invariant of the pair $(K, \pi_1(X_K) \to \ZZ)$.
This last picture connects to Reidemeister torsion.

Why is this good for geometric topology?

\begin{dfn}
  Say $K \subset S^3$ is \emph{fibered} if there exists a fibration $S^3 \setminus K \to S^1$ with fibers given by Seifert surfaces.
\end{dfn}

\begin{ex}
  The unknot is fibered.
\end{ex}

\begin{ex}
  The trefoil is also fibered, though this is harder to see.
\end{ex}

\begin{thm}
  If a knot $K$ is fibered, then $\Delta_K(t)$ is monic and has degree equal to $2 g(F)$, where $g(F)$ is the genus of the fiber.
\end{thm}

\subsection{Some representation theory of $\sl_2(k)$}

Recall that the Lie algebra $\sl_2(k)$ has basis $H, E, F$ with $EF - FE = H$.

If $\charop k = 0$, then $\sl_2(k)$ is semisimple, and there are simple representations $V_m$ for $m \in \NN$.
Furthermore, the weights live in $\ZZ$.

Things are more complicated when $\charop k = p > 0$.
Here we still have simple modules $V_0, V_1, \dots, V_{p-2}$.
However, there are infinitely many simple modules $V_\alpha$ in dimension $p$, and they are parametrized by $\alpha \in k$.
The vector space $V_\alpha$ has basis $v_0, \dots, v_{p-1}$, and $H$ acts by $H v_i = (\alpha - 2i) v_i$.
The elements $E$ and $F$ still act by raising / lowering degree.

We're especially interested in the quantum world.
Here we work with a Hopf algebra $U_q \sl_2$ (over $\CC$) defined using a parameter $q$.
This has generators $H, E, F$ satisfying
\[
  EF - FE = \frac{q^H - q^{-H}}{q - q^{-1}}.
\]
When $q$ is not a root of $1$, the theory behaves much like that of $\sl_2(\CC)$.
When $q^{2p} = 1$, the theory behaves more like that in characteristic $p$.
In particular, we have simple modules $V_\alpha$ for $\alpha \in \CC$.
The coefficients in the raising / lowering operators are controlled by quantum numbers $[\alpha + i]_q$.

\subsection{Quantum invariants of knots}

Let $V$ be a finite-dimensional representation of $U_q \sl_2$.
There exists a braiding $c: V \otimes V \to V \otimes V$ satisfying the Yang-Baxter equation
\[
  (1 \otimes c) \circ (c \otimes 1) \circ (1 \otimes c) = (c \otimes 1) \circ (1 \otimes c) \circ (c \otimes 1).
\]
This gives knot invariants by writing a knot as a braid closure and viewing the braid as a string diagram in $\Rep U_q \sl_2$.

For generic $q$, we get invariants $I_{V_m}(K) \in \ZZ[q^{\pm 1}]$.
These are the \emph{colored Jones polynomials}.

If $q = \zeta$ for $\zeta$ a primitive $2p$th root of unity, we have more representations $V_\alpha$.
Thus we get $I_{V_\alpha}(K) \in \ZZ[\zeta][t^{\pm 1}]$ (where $t = q^\alpha$).
We may call these $\ADO_p(K, t)$ (after Akutsu--Deguchi--Ohtsuki).
Note that $\ADO_2(K, t) = \Delta_K(t)$.

\begin{thm}[LN-vdV 2024]
  Let $K$ be a fibered knot.
  Then $\ADO_p(K, t)$ is $q$-monic, and $\deg_t \ADO_p(K) = 2 (p-1) g(F)$.
\end{thm}

The proof uses some difficult results in contact geometry -- the speaker doesn't know if it's possible to give an algebraic topology proof.
There's a theorem of Stallings that says that, if $K$ is fibered, then $[\pi_1, \pi_1]$ is finitely generated.

\subsection{$G$-equivariance}

If $q$ is generic, then $\Rep U_q \sl_2$ is a tensor category.
If $q$ is a root of $1$, then we may upgrade this structure to view $\Rep U_q \sl_2$ as a $\CC / 2 \ZZ$-tensor category.

Turaev tells us that we should think of this in terms of $G$-crossed braided tensor categories.
From the $G$-crossed braided story, we get invariants of pairs $(K, \rho: \pi_1(X_K) \to G)$.
Thus, when $q$ is a root of unity, we can generalize the ADO invariants to invariants of knots $K$ together with $\rho: \pi_1(X_K) \to \CC / 2\ZZ$.
There are many interesting questions about interpretations of these.

\section{Ralph Kaufmann -- Orbifolding TFTs and CFTs}

This was a slide talk so my notes are pretty heuristic.

There is a notion of $G$-Frobenius algebra ``equivariantizing'' the usual Frobenius algebras.
We may categorify this to get $G$-fusion categories.

Note that $G$-Frobenius algebras are modules over $D(k[G])$, the Drinfeld double of the group ring $k[G]$.
These may also be understood as functors out of $\Lambda G$.

We like to twist by characters $\chi: G \to \CC^\times$ to account for the physics of orbifold LG modules.
More generally, if $\Cc$ is a monoidal category and $\II_\chi$ is an even invertible element, we may twist actions by $\chi$.
By this point I had stopped being able to follow things, sorry!

\section{Renee Hoekzema -- Invertible and Striped Cylinder TQFTs}

This was also a slide talk based on joint work with many collaborators.

\subsection{Invertible field theories}

Recall that 2d TQFTs correspond to commutative Frobenius algebras.
To classify more general TQFTs, we can either:
\begin{itemize}
  \item Find generators-and-relations descriptions of cobordism categories.
  \item Restrict to nice classes of TQFTs, e.g.\ invertible TQFTs.
\end{itemize}

Recall that a TQFT is invertible if every object and morphism in $\Cob_n$ is sent to an invertible object.
Freed and Hopkins conjectured that invertible TQFTs classify symmetry-protected topological phases of matter.
Furthermore, one can classify invertible TQFTS valued in $\sVect$ -- they correspond to ``SKK invariants,'' i.e.\ homomorphisms $\SKK_n \to \CC^\times$.
Here $\SKK_n$ is a certain quotient of the Grothendieck group of $n$-manifolds by saying that 
\[
  [M \cup_\phi N] - [M \cup_\psi N] = [M' \cup_\phi N'] - [M' \cup_\psi N'].
\]
whenever we have the same gluing data $\phi, \psi$.
The proof appears in future work of Kreck--Stolz--Teichner.

One may fit $\SKK_n$ into short exact sequences involving $\Omega_n$.

\subsection{Tangential structures}

\begin{dfn}
  Fix a map $\xi: B_n \to \BO_n$.
  A \emph{$\xi$-structure} on a manifold $M$ is a lift of the tangent classifier $M \to \BO_n$.
\end{dfn}

We may fit the corresponding groups into an ``SKK sequence:''
\[
  \begin{tikzcd}
    0 \rar & \angles{S^n}_{\SKK^\xi} \rar & \SKK^\xi_n \rar & \Omega^\xi_n \rar & 0.
  \end{tikzcd}
\]

\begin{qn}
  How do we compute $\angles{S^n}_{\SKK^\xi}$?
  It's always $0$, $\ZZ/2$, or $\ZZ$ (I might have written this wrong).
\end{qn}

\begin{qn}
  For which $\xi$ do these sequences split?
\end{qn}

Ebert showed the sequence splits for $\BSO_n \to \BO_n$ when $n \equiv 1 \pmod 4$.
The speaker showed many results for other parts of the Whitehead tower of $\BO_n$.

\begin{conj}
  If $n$ is odd then the SKK sequence always splits.
\end{conj}

\subsection{Generators and relations}

Let's go back to the question of understanding a generators-and-relations presentation of $\Cob_n$.
Consider nested manifolds $M_I$ consisting of $M^{d_1} \subset M^{d_2} \subset \dots \subset M^{d_k}$.
We may define corresponding nested cobordisms and a nested cobordism category $\Cob_I$.
The generating morphisms for $\Cob_I$ are given by elementary cobordisms with $0$ or $1$ Morse critical points.
Relations are harder and are the subject of ongoing work.

What does this mean in terms of algebra, i.e.\ what classifies a symmetric monoidal functor $\Cob_I \to \Vect$?
Within $\Cob_I$ for $I = \{0, 1, 2\}$, one may define a ``cylinder category'' $\Cyl$.
Symmetric monoidal functors $\Cyl \to \Cc$ may be described explicitly.
They give rise to both affine and annular Temperley--Lieb algebras.

One may also relate $\Cyl$ to the cyclic category $\Lambda\op$ (a modification of the simplex category used for cyclic homology).
There's an inclusion $\Lambda\op \hookrightarrow \Cyl_\mathrm{even}$.
There's a ``cylinder bar construction'' one can use to capture the structure of $\Cyl$.

\section{Problem Session}

Here are a few natural problems arising from the work today:
\begin{itemize}
  \item (Eric Samperton) Consider the characters $\chi: G \to \CC^\times$ appearing in RK's twisted Frobenius algebras.
    What does this mean from the TQFT perspective?
    RK says it's something like a ``Tate object'' changing the fusion rules in the fusion categories.
    How does this fit in?
  \item (Kyle Kawagoe) How do we equivariantize RH's $\Cyl$ category?
  \item (David Green) Does $\Cyl$ have a (symmetric) monoidal structure?
  \item (John Nolan) Can we understand the $G$-crossed invariants in DLN's talk in terms of local systems / algebraic topology?
  \item (Daniel L\'opez Neumann) Let $q$ be fixed and let $n$ grow.
    Is there a classical interpretation of the knot invariants associated with $U_q \sl_n$?
  \item (Daniel L\'opez Neumann) More generally, ``Alexander theorems'' are theorems about non-semisimple quantum invariants.
    What is the relationship between the SQP property for knots and having positive powers of $q$ in the polynomial?
    (This only works for generic coefficients.)
  \item (Ralph Kaufmann) Conjecture about Green--Plesser mirror symmetry in terms of (de-)equivariantization of matrix factorization categories and Hochschild homology.
  \item (Karl Dovermann) Does (strong) algebraic realization for $G_2$ imply (strong) algebraic realization for $G$?
  \item (Kyle Kawagoe) How does realization work for infinite compact Lie groups? e.g. $\ZZ_2 \ltimes \Urm(1)$?
  \item (Renee Hoekzema) Does there exist a 4-orientable manifold with odd Euler characteristic?
    The speaker has some specific examples in mind (e.g.\ Rosenfeld projective planes).
  \item (John Nolan) Is there anything interesting we can say about the SKK invariants for the Postnikov tower for $\BO_n$ (as opposed to the Whitehead tower)?
  \item (Eric Samperton) What is an equivariant TQFT?
    Want to understand the difference between TQFTs for $X$ with a principal $G$-bundle and TQFTs for $X$ with a $G$-action.
    Is there a sensible cobordism category for the latter?
  \item (Kyle Kawagoe) What about replacing $\Vect$ (in the codomain of a TQFT) by $\Rep(G)$?
    How do these interact?
\end{itemize}

\section{Robert Oeckl -- Compositional QFT: Motivation and Axiomatics}

This is joint work with Juan Orendain (speaking next on the same subject).
The goal is to use ideas from TQFT to understand general physics.

\subsection{Review of quantum theory}

States live in a Hilbert space $\Hc$.
The dynamics of a quantim system are given by a time evolution operator $U_{[t_1, t_2]}: \Hc \to \Hc$ or equivalently a transition amplitude $\ip{\psi_2}{U_{[t_1, t_2]} \psi_1}$.
In QFT the transition amplitudes may be computed using Feynman intervals.
The spacetime composition property of path integrals gives rise to gluing in TQFT: $U_{M \cup N} = U_N \circ U_M$.

Realistic QFT is hard to formalize for the following reasons:
\begin{itemize}
  \item Directionality of cobordisms in TQFT can be identified with time evolution.
    This is too restrictive for real QFT because the spacetime topology is boring.
    In particular, decomposing along spacelike hypersurfaces doesn't tell us much!
    We want to be able to decompose spacetime into compact regions, so that we can take limits as we work with finer and finer decompositions.
    This forces us to work with an extended setting (including corners, \dots).
  \item The vector spaces appearing in QFT require an inner product.
  \item Real QFT requires infinite-dimensional vector spaces.
  \item TQFT doesn't make physical predictions.
    To make predictions in real QFT, we use the S-matrix which relates transition probabilities between spacelike hypersurfaces.
\end{itemize}

General boundary QFT avoids some of the usual problems with TQFT-type approaches.
It can be used to formalize loop quantum gravity / spin foams.
However, GBQFT is less elegant than TQFT, and it introduced many \emph{ad hoc} structures.

\subsection{Compositional QFT}

Compositional QFT tries to incorporate the best aspects of GBQFT and TQFT.
In particular, CQFT has:
\begin{itemize}
  \item Better descriptions of \emph{gluing}.
  \item Notions of spacetime symmetries and equivariance.
  \item Functoriality related to involutive symmetric monoidal categories.
\end{itemize}

Let's consider topological, oriented, compact manifolds of a fixed dimension.
Say that $f: \Sigma \to \Sigma'$ is a gluing function (for hypersurfaces $\Sigma, \Sigma'$) if $f$ is continuous, closed, surjective, and orientation-preserving, and $f$ restricts to an embedding on the interior of $\Sigma$.
One may also make sense of ``relative gluing diagrams'' in codimension $0$.
Such diagrams let us handle examples such as gluing the sides of a square to form a cylinder.

The target category of a compositional QFT is $\sVect$ with involution given by $V \mapsto \ol{\Pi V}$.
We send spacelike hypersurfaces to state spaces and gluing functions to time evolution maps.
The above axioms give an inner product on the state spaces of our QFTs as well as other useful structures.

\section{Juan Orendain -- Compositional QFT and Isomorphism Extension}

Note that CQFT aims to do QFT, not compute manifold invariants.
Gluing is handled through ``relative gluing diagrams'' inspired by applied category theory.
This gives CQFT a ``built in equivariance.''

\subsection{The idea behind CQFT}

Let $X_1$ and $X_2$ be regions in spacetime.
We would like to understand physics by gluing the behavior of $X_1$ and $X_2$ together.
CQFT assigns state spaces to boundaries of regions and more generally to hypersurfaces.

In the version sketched above, we let regions be compact topological manifolds of dimension $n + 1$.
We also consider hypersurfaces in these regions.
Gluing is handled by taking colimits, and in CQFT we choose \emph{explicit colimit representations}.

Note that gluing functions are stable under compositions, changes of orientation, disjoint unions, and homeomorphisms.
Thus we get an (involutive) category of gluing functions.
Note that this is not a dagger category, nor is it a compact closed category.
The involution does not change the order of morphisms in the category!

Let $\Mc_{\mathrm{hyp}}^{\mathrm{gl}}$ be the category of hypersurfaces and gluing functions.
This is an involutive symmetric monoidal category.
Similarly, $\Vect_\CC$ is an involutive symmetric monoidal category (with $J(V) = \ol{V}$).

The relative gluing diagrams mentioned before let us glue regions along \emph{parts} of their boundaries (not necessarily the entire boundaries).
We may also organize relative gluing diagrams into an involutive symmetric monoidal category.

The codimension $1$ part of a CQFT is equivalent to the data of a (strictly) involutive strong symmetric monoidal functor $\Mc_{\mathrm{hyp}}^{\mathrm{gl}} \to \sVect_\CC$.
More generally, we may consider involutive SMCs $\Cc$ fibered over $\Mc_{hyp}^{gl}$ and consider nice functors $\Cc \to \sVect$ (or some similar category).
However, this doesn't encompass the behavior of CQFTs on regions.

\section{Cesar Galindo -- Minimal Nondegenerate Extensions of Braided Fusion Categories}

\subsection{Review of fusion categories}

Recall that a \emph{fusion category} (over $k = \CC$) is:
\begin{itemize}
  \item Semisimple, with finitely many iso classes of simple objects,
  \item Monoidal and rigid (i.e.\ every object is dualizable), with $\onebb$ simple
\end{itemize}

\begin{ex}
  Let $G$ be a finite group.
  Then $\Rep(G)$ is a fusion category.
\end{ex}

\begin{ex}
  Let $G$ be a finite group.
  Then the category $\Vect_G$ of $G$-graded vector spaces is a fusion category (with tensor product given by $\CC_g \otimes \CC_h = \CC_{gh})$.
\end{ex}

\begin{ex}
  For a semisimple Lie algebra $\gfr$ and $k \in \NN$, there is a fusion category $\Cc(\gfr, \NN)$.
\end{ex}

\subsection{Grothendieck rings and Frobenius--Perron dimensions}

We can understand fusion categories by decategorifying and looking at the Grothendieck ring $K_0(\Cc)$.
We can write
\[
  [X] \otimes [Y] = \sum_{[Z]} N_{X,Y}^{Z} [Z]
\]
where $N_{X,Y}^Z = \dim_\CC \Hom(Z, X \otimes Y)$.

\begin{ex}
  The Grothendieck ring $K_0(\Rep(G))$ is the character ring of $G$.
\end{ex}

\begin{ex}
  The Grothendieck ring $K_0(\Vect_G)$ is $\ZZ[G]$.
\end{ex}

\begin{ex}
  There is a Fibonacci fusion category with fusion ring generated by $\onebb$, $\tau$ satisfying $\tau^2 = \onebb + \tau$.
\end{ex}

We may define a \emph{Frobenius--Perron dimension} $\FPdim: K_0(\Cc) \to \RR$.
This is the unique homomorphism satisfying $\FPdim(X) > 0$ for all nonzero $X$.
We let $\FPdim(\Cc) = \sum_{X \in \Irr(\Cc)} \FPdim(X)^2$.

\subsection{Braidings}

A \emph{braided fusion category} is a fusion category together with a braiding $c_{X,Y}: X \otimes Y \to Y \otimes X$ satisfying the usual axioms.

\begin{ex}
  We may view $\Rep(G)$ as a braided fusion category with the trivial braiding $c(v \otimes w) = w \otimes v$.
  We call $\Rep(G)$ a \emph{Tannakian category}.
\end{ex}

\begin{ex}
  Let $\tilde{G}$ be a finite group and fix $z \in Z(\tilde{G})$ with $z^2 = 1$.
  Then we may decompose $\Rep(\tilde{G}) = \Rep(\tilde{G})_0 \oplus \Rep(\tilde{G})_1$ according to the sign of the action of $z$.
  We may define a braiding on this category by $c(v \otimes w) = (-1)^{|v| |w|} w \otimes v$.
  We call the resulting braided fusion category $\Rep(\tilde{G}, z)$ and think of $(\tilde{G}, z)$ as a supergroup (so $\Rep(\tilde{G}, z)$ is a \emph{super-Tannakian category}).
  For $(\tilde{G}, z) = (\ZZ_2, [1])$ we get $\Rep(\ZZ_2, \tilde{1}) = \sVect$.
\end{ex}

\begin{ex}
  If $\Cc$ is a fusion category, we may define a \emph{Drinfeld center} $Z(\Cc)$.
  This is a braided fusion category.
  Objects of $Z(\Cc)$ are pairs $(X, c_{X,-})$ where $c_{X,-}$ satisfies the relevant part of the associativity axiom for braidings.
  There is a natural map $Z(\Cc) \to \Cc$, and we have $\FPdim(Z(\Cc)) = \FPdim(\Cc)^2$.
\end{ex}

Within a braided fusion category $\Bc$, we may define a \emph{M\"uger center} $Z_2(\Bc)$ as the full subcategory of $\Bc$ consisting of objects $X$ such that the braiding $c_{X,-}$ is symmetric.
We say $\Bc$ is:
\begin{itemize}
  \item \emph{Symmetric} if $Z_2(\Bc) = \Bc$.
    The symmetric fusion categories were classified by Deligne -- they are all Tannakian or super-Tannakian.
  \item \emph{Nondegenerate} if $Z_2(\Bc) = \Vect$.
  \item \emph{Slightly nondegenerate} if $Z_2(\Bc) = \sVect$.
\end{itemize}

\subsection{Extensions of fusion categories}

\begin{dfn}
  An extension of braided fusion categories $\Cc \supset \Bc$ is \emph{minimal} if:
  \begin{itemize}
    \item $\Cc$ is nondegenerate.
    \item $\FPdim(\Bc) \FPdim(Z_2(\Bc)) = \FPdim(\Cc)$.
  \end{itemize}
\end{dfn}

There are two natural questions:
\begin{itemize}
  \item When does a minimal extension exist?
  \item How do we construct the set of minimal extensions $\Mext(\Bc)$ up to equivalence?
\end{itemize}

\begin{thm}[Lan--Kong--Wen]
  \begin{itemize}
    \item If $\Ec$ is symmetric, then $\Mext(\Ec)$ is an abelian group.
    \item For any braided fusion category $\Bc$, $\Mext(\Bc)$ is a $\Mext(Z_2(\Bc))$-torsor.
  \end{itemize}
\end{thm}

To understand existence of minimal extensions of $\Bc$, we split up according to whether $Z_2(\Bc)$ is Tannakian or super-Tannakian.

\subsection{Extensions in the Tannakian case}

If $Z_2(\Bc)$ is Tannakian, say we have $\Rep(G) \subset \Bc$.
Let $A = \Osc(G) \in \Rep(G)$, and let $\Dc = \Bc_G = \Mod_A(\Bc)$.
There is a natural braided monoidal structure on $\Dc$ given by $\otimes_A$.
Because $G$ acts on $A$, we get an action of $G$ on $(\Dc, \otimes_A)$, i.e.\ a functor $G \to \Aut_\otimes(\Dc)$.
By \emph{equivariantization}, we may recover $\Bc$ from $\Dc$ with its $G$-action via $\Bc = \Dc^G$.
That is, $\Bc = (\Bc_G)^G$.
We may also compute $\FPdim(\Bc_G) = \FPdim(\Bc) / |G|$.

The above allows us to view $\Dc$ as a \emph{braided $G$-crossed fusion category}.
That is, $\Dc = \oplus_{g \in G} \Dc_g$, and we have $\Dc_g \otimes \Dc_h = \Dc_{gh}$ and ${}^g \Dc_h \subset \Dc_{g h g\inv}$.
There is also a braiding, but now it goes from $\Dc_g \otimes \Dc_h$ to ${}^g \Dc_h \otimes \Dc_g$.

There's an equivalence between braided $G$-crossed fusion categories $\Dc$ with $\Dc_e = \Sc$ and homomorphisms of monoidal $2$-categories $G \to \Mod(\Sc)^\times$.
If $\Sc$ is nondegenerate, then $\Mod(\Sc)^\times = \Aut_\otimes^{\mathrm{br}}(\Sc)$.
This can be used to compute obstructions: a homomorphism $\tilde{\rho}: G \to \Aut_\otimes^{\mathrm{br}}(\Sc)$ determines an obstruction $o_4(\tilde{\rho}) \in H^4(G, \CC^\times)$.

In particular, if $\Bc$ is a braided fusion category with $Z_2(\Bc)$ Tannakian, the map $G \to \Aut_\otimes^{\mathrm{br}}(\Sc)$ determines a class $o_4(\Bc) \in H^4(G, \CC^\times)$.

\begin{thm}[G--Venargas--Ramirez]
  If $\Bc$ is modularizable, then $\Bc$ admits a minimal extension if and only if $o_4(\Bc) = 0$.
\end{thm}

\subsection{Extensions in the super-Tannakian case}

If $Z_2(\Bc)$ is super-Tannakian, we can still find a maximal Tannakian subcategory $\Rep(G) \subset \Rep(\tilde{G}, z) \subset \Bc$.
\begin{thm}
  Let $\Bc$ be a braided fusion category with $Z_2(\Bc) = \Rep(\tilde{G}, z)$, then $\Bc$ admits a minimal extension if and only if
  there is a minimal extension $(\Bc_G)_e \subset \Sc$ and a fermionic action of $(\tilde{G}, z)$ on $\Sc$ such that
  $\Bc_G \subset \Sc$, $\Bc_G$ is $G$-stable, $(\Bc_G)^G \simeq \Bc$, and $\Oc(S^G) = 0$.
\end{thm}

\section{Dmitri Nikshych -- Localization, Equivariantization, and Group-Theoretical Invariants of Braided Fusion Categories}

This was a slide talk so my notes are sketchy.

\subsection{More examples}

Here are some more examples of braided fusion categories:

\begin{ex}
  Let $A$ be an abelian group and let $\phi: A \to k^\times$ be a quadratic form.
  Then $\Vect_A$ has a unique braiding such that $c_{a,a} = \phi(a)$ for all $a \in A$.
  We write the corresponding braided fusion category as $\Cc(A, q)$.
  This is the most general example of a pointed braided fusion category.
  The Drinfeld center $Z(\Vect_A)$ is $\Cc(A \oplus \hat{A}, h)$ where $h$ is the natural ``hyperbolic form.
\end{ex}

\begin{ex}
  $R$-matrices on Hopf algebras also give braided fusion categories.
\end{ex}

\subsection{Centralizers}

Let $\Cc$ be a BFC.
The centralizer of a set of objects $\Sc$ is
\[
  \Sc' = \bset{X \in \Cc}{c_{XY} c_{YX} = \id_{X \otimes Y} \forall Y \in \Sc}.
\]
This is best thought of as analogous to an orthogonal complement in linear algebra.
We say that $\Cc$ is nondegenerate if $\Cc' = \Vect$.

\subsection{Localization and gauging}

Let $\Bc$ be a braided fusion category, and let $\Ec = \Rep(G) \subset \Bc$ be a Tannakian subcategory.
We define the \emph{localization} of $\Bc$ with respect to $\Ec$ as $\Bc^\loc = \Ec' \boxtimes_\Ec \Vect$.
Equivalently, let $A = \Oc(G) \in \Rep(G)$, and let $\Bc^\loc$ be the category of \emph{local} $A$-modules in $\Bc$, i.e.\ those such that the module action plays well with the braiding.
There is a natural action $G \to \Aut^\br(\Bc^\loc)$ (really we can think of this as a monoidal $2$-functor).

\begin{ex}
  Let $\Bc = \Rep(D(G))$, and let $N$ be a normal subgroup of $G$.
  Then $\Ec = \Rep(G / N) \subset \Bc$ is Tannakian, and $\Bc^\loc = \Rep(D(N))$.
  The group $G / N$ acts on $\Ec$ (the action is induced by the conjugation action $G \curvearrowright N$).
\end{ex}

Gauging is the inverse problem of localization.
Given a braided fusion $\Bc$ with an action of a group $G$, we want to construct $\Dc \supset \Ec = \Rep(G)$ with $\Bc = \Ec' \boxtimes_\Ec \Vect$.
Call this $\Dc$ a \emph{$G$-gauging} of $\Bc$.
These are typically not unique if they exist: they are parametrized by monoidal $2$-functors $G \to \Mod(\Bc)^\times$.

\begin{ex}
  $G$-gaugings of $\Vect$ are all of the form $\Rep(D^\omega(G))$.
\end{ex}

To construct a $G$-gauging from a monoidal $2$-functor $F: G \to \Mod(\Bc)^\times$, we let $\Cc_g = F(g)$.
Then $\Cc = \oplus_{g \in G} \Cc_g$ is a $G$-crossed braided category with $\Cc_e = \Bc$.
The corresponding $G$-gauging of $\Bc$ is $\Dc = \Cc^G$.

\begin{thm}[Etingof--N--Ostrik]
  $\Aut^\br(\Bc) = \Mod(\Bc)^\times$ as categorical groups.
\end{thm}

This lets us construct monoidal $2$-functors as follows:
\begin{itemize}
  \item Find a homomorphism $G \to \Aut^\br(\Bc)$.
  \item Lift it to a monoidal functor / categorical action.
  \item Lift the monoidal functor to a monoidal $2$-functor.
    (This involves an obstruction on $H^4(G, k^\times)$.)
\end{itemize}

\subsection{Cores}

Assume from now on that $\Bc$ is nondegenerate.
Let $M(\Bc)$ be the set of \emph{maximal} Tannakian subcategories of $\Bc$.

\begin{thm}[Drinfeld--Gelaki--N--Ostrik]
  If $\Ec_1, \Ec_2 \in M(\Bc)$, then $\Ec_1' \boxtimes_{\Ec_1} \Vect = \Ec_2' \boxtimes_{\Ec_2} \Vect$.
\end{thm}

The corresponding localization is called the \emph{core} of $\Bc$.
Note that $\Ec_1$ and $\Ec_2$ need not be $\Rep(G)$ for the same group $G$.

\begin{ex}
  We have $\Rep D(\ZZ_4) = \Rep D^\omega(\ZZ_2 \times \ZZ_2)$.
  In particular, descriptions of categories as twisted Drinfeld doubles are not unique.
\end{ex}

How can we get relevant invariants?

\subsection{Tannakian radicals}

Let $\Rad(\Bc) = \cap_{\Ec \in M(\Bc)} \Ec$.
This intersection is Tannakian, so $\Rad(\Bc) = \Rep(G_\Bc)$ for some group $G_\Bc$.
This is an invariant of $\Bc$.
Say that $\Bc$ is reductive if $\Rad(\Bc) = \Vect$.

The localization corresponding to $\Rad(\Bc)$ is the \emph{mantle} $\Mantle(\Bc)$ of $\Bc$.
The terminology is inspired by geology: there are localizations $\Bc \to \Mantle(\Bc) \to \Core(\Bc)$.

\begin{thm}
  The mantle of $\Bc$ is reductive.
\end{thm}

Thus $\Bc$ is determined by a monoidal $2$-functor $G_\Bc \to \Mod(\Mantle(\Bc))^\times$.

\begin{ex}
  Let $G$ be a finite group.
  Then $\Rad(\Rep(D(G))) = \Rep(G / N(G))$ where $N(G) = \bangle{A}{A \textrm{ is normal in $G$ and abelian}}$.
  We have $\Mantle(\Rep(D(G))) = \Rep(D(N(G)))$.
\end{ex}

Say a fusion category $\Cc$ is \emph{nilpotent} if it is obtained from $\Vect$ by a sequence of graded extensions.
Say that $\Cc$ is \emph{weakly group-theoretical} if it is Morita dual to a nilpotent one.
All known examples of fusion categories with integral FP dimension are weakly group-theoretical.

\begin{thm}
  Let $\Bc$ be a non-degenerate BFC.
  Then $\Bc$ is weakly group-theoretical if and only if $\Mantle(\Bc)$ is nilpotent.
\end{thm}

Note that if $\Bc$ is reductive then $\Bc \cap \Bc'$ is either $\Vect$ or $\sVect$.
Let $\Tc(\Bc)$ be the subcategory of $\Bc$ generated by all Tannakian subcategories of $\Bc$.
Then $\Tc(\Bc)$ is nilpotent, and $\Tc(\Bc)'$ is anisotropic (i.e.\ it has no Tannakian subcategories other than $\Vect$).
Both $\Tc(\Bc)$ and $\Tc(\Bc)'$ are reductive.

\subsection{Applications to classification of BFCs}

Let's fix a given Frobenius--Perron dimension $N$.

\begin{enumerate}
  \item Look for pairs $(\Bc, G)$ with $\Bc$ reductive, $G$ a finite group, and $\FPdim(\Bc)|G|^2 = N$.
  \item List all monoidal functors $G \to \Aut^\br(\Bc)$.
  \item Determine all lifts of these to monoidal $2$-functors $G \to \Mod(\Bc)^\times$.
    These give $G$-gaugings of $\Bc$.
  \item Select gaugings $\tilde{\Bc}$ for which $\Rad(\tilde{\Bc}) = \Rep(G)$.
\end{enumerate}

We may also use this to enumerate non-equivalent centers.

\subsection{Further localization of nilpotent parts}

Let $\Bc$ be nilpotent and braided.
We have a Sylow decomposition $\Bc = \boxtimes_p \Bc_p$.
We may define an inductive sequence of localizations to reduce the study of nilpotent BFCs to the examples coming from quadratic forms.

\section{Alexei Davydov -- Homotopy groups of the group of invertibles in \texorpdfstring{$d\Rep(G)$}{dRep(G)}}

\subsection{The idea of TQFT}

Suppose we have a closed $d$-manifold invariant $\tau(M) \in k$.
We'd like to try to organize this into a TQFT.

To get state spaces for $(d-1)$-manifolds $\Sigma$, we can let $V(\Sigma) = \bangle{N}{\partial N = \Sigma}$.
Then we have a pairing $V(\Sigma) \times V(\Sigma) \to k$ given by $(N, N') \mapsto \tau(N \cup_\Sigma N')$.
We let $Z(\Sigma) = V(\Sigma) / \ker$.

We may hope that this extends to a symmetric monoidal functor $\Cob_{d-1,d} \to \Vect$.
The ``quantum'' part of TQFT corresponds to the symmetric monoidal property $Z(X \sqcup Y) = Z(X) \otimes Z(Y)$.

We might try to extend this even further, sending $(d-2)$-manifolds to ``$2$-vector spaces'' (i.e.\ (nice) linear categories), etc.
This gives an extended TQFT.
We'd like to think of the higher (symmetric monoidal) category $d\Vect$ of $d$-vector spaces as the trivial $(d+1)$-vector space.
We could try to use this as an inductive definition of $d\Vect$.
Properly formalizing this requires the use of higher categories.

\subsection{Homotopy groups of $(d\Vect)^\times$}

Let $(d\Vect)^\times$ be the $d$-categorical group of invertibles. 
We may think of this as a homotopy type, so we may discuss the homotopy groups $\pi_i((d\Vect)^\times)$.
Here $\pi_0(-)$ is the group of isomorphism classes of objects, and $\pi_i(-) = \pi_0(\Omega^i - )$.
In particular, $\pi_1(-)$ is the group of isomorphism classes of $1$-morphisms $\onebb \to \onebb$, etc.
Note that $d\Vect = \Omega (d+1)\Vect$.
These assemble into a ``categorical spectrum.''

For different values of $d$, we get:
\begin{itemize}
  \item $\pi_0((0\Vect)^\times) = k^\times$.
  \item $\pi_0((1\Vect)^\times) = \{ 1 \}$.
  \item $\pi_0((2\Vect)^\times) = \Br(k)$, the \emph{Brauer group}, consisting of Azumaya (i.e.\ Morita-invertible) algebras.
    When $k$ is algebraically closed of characteristic zero, we have $\Br(k) = \{ 1 \}$.
  \item $\pi_0((3\Vect)^\times)$ is the group of Azumaya fusion categories.
    When $k$ is algebraically closed of characteristic zero, this is still $\{ 1 \}$.
  \item $\pi_0((4\Vect)^\times) = W$, the Witt group of invertible nondegenerate braided fusion categories (a countable infinite group).
\end{itemize}

To understand the computation $\pi_0((4\Vect)^\times) = W$, let's first understand what happens when we replace $\Vect$ by $\sVect$.
Here we have:
\begin{itemize}
  \item $\pi_0((0\sVect)^\times) = k^\times$.
  \item $\pi_0((1\Vect)^\times) = \ZZ_2$.
  \item $\pi_0((2\Vect)^\times) = \ZZ_2$ (given by the first Clifford algebra).
  \item $\pi_0((3\Vect)^\times) = \{ 1 \}$.
  \item $\pi_0((4\Vect)^\times) = \sW$, the super-Witt group.
    Here $\sW = \sW_\pt \oplus \sW' = \ZZ^\infty \oplus \ZZ_2^\infty \oplus \ZZ_4^\infty$.
\end{itemize}

Computations in the super case are actually easier than those in the classical case due to M\"uger factorizations.
One may compute $W$ from $\sW$ using $\ZZ/16 \to W \to \sW$ (related to Kitaev's ``16-fold way'').
We end up looking at a cofiber sequence of categories  relating $\pi_i((d\Vect)^\times)$ and $\pi_i((d\sVect)^\times)$.

\subsection{Equivariant version}

Let $G$ act on $d$-vector spaces.
We look at $\Rep_{d\Vect}(G) =: d\Rep(G)$.
As above, we iteratively construct $\Mod((d-1)\Rep(G))$.

We may extend scalars to consider super-representations of $G$.
There's a natural split homomorphism $d\Vect \to d\Rep(G)$ which we may use to compute $\pi_0(d\Rep(G))^\times$.
We end up with $\pi_i(d\Rep(G)) = H^i(G, k^\times)$ for $i \leq 3$.
The connecting homomorphisms in the corresponding cofiber sequence are often isomorphisms.
Note that things break down in higher degrees though!

\section{K\"ur\c{s}at S\"ozer -- From Homotopy 0-, 1-, and 2-Types to Graded Fusion Categories}

Slide talk.
This is based on joint work with Alexis Virelizier.
There was a lot of standard review of topology which I didn't write down.

\subsection{Crossed modules}

Connected homotopy 1-types are all $K(G, 1)$'s, i.e.\ groups model connected homotopy 1-types.
Connected homotopy 2-types are modeled by ``crossed modules'' (by a theorem of Mac Lane--Whitehead).
Crossed modules are certain homomorphisms $\chi: E \to H$ where $H \curvearrowright E$ and the homomorphism is compatible with the action (in a precise sense).
The corresponding homotopy 1-type $B\chi$ has $\pi_1(B\chi) = \coker(\chi)$ and $\pi_2(B\chi) = \ker(\chi)$.

Here are a few examples:
\begin{itemize}
  \item Any group $G$ gives a natural crossed module $1 \to G$.
  \item Any abelian group $A$ gives a natural crossed module $A \to 1$.
  \item If $E \subset H$ is a normal subgroup, then $E \to H$ is a crossed module.
  \item If $(X, A)$ is a pair of pointed topological spaces, then $\pi_2(X, A) \to \pi_1(A)$ is a crossed module.
\end{itemize}

Crossed modules capture the same information as strict 2-groups.

\subsection{Graded fusion categories}

The idea here is that:
\begin{itemize}
  \item Fusion categories are like $0$-types.
  \item Group graded fusion categories are like $1$-types.
  \item Crossed module graded fusion categories are like $2$-types.
\end{itemize}

For a crossed module $\chi: E \to H$, one may come up with a notion of $\chi$-graded fusion categories.
Such a category $\Cc$ has $H$-grading on objects and $E$-grading on $\Hom$s.
It's also possible to come up with notions of pivotal and spherical structures.
There are many subtleties regarding notions of semisimplicity here!

\begin{ex}
  One can construct $\chi$-graded fusion categories from strict $2$-groups.
\end{ex}

It's also possible to define ``Hopf $\chi$-coalgebras.''
The corresponding module categories give examples of $\chi$-graded fusion categories (in nice cases).

The reason we care is that we might want to consider $\sigma$-models with target given by a connected homotopy $2$-type.
This encompasses versions of homotopy quantum field theories and extensions of Kuperberg invariants.

\section{Problem Session}

\begin{itemize}
  \item (Nico Bridges / K\"ur\c{s}at S\"ozer) What does a map to $B\chi$ classify?
  \item (Eric Samperton) Is there an analogue of the Eilenberg--Mac Lane isomorphism $\Quad(A) = H^3_\ab(A; \Urm(1))$ for pointed $G$-crossed modular fusion categories?
    What about the special case where $G$ is abelian?
  \item (Shawn Cui) Is there a non-semisimple version of $G$-crossed or $\chi$-crossed modular tensor categories?
  \item (K\"ur\c{s}at S\"ozer) What are the homotopy fixed points of the categories appearing in AD's talk?
    Are $\pi_0(d\Rep(G))$ the homotopy $G$-fixed points?
  \item (Kyle Kawagoe) How do the sequences in AD's talk fit into ``$n$-fold ways?''
    Can we find explicit representatives?
  \item (H\'ector Mart\'in Pe\~na Pollastri) Can we universally turn fusion categories into $\chi$-graded fusion categories?
  \item (John Nolan) Can we do homotopy $n$-type gradings for fusion categories / analogues?
  \item (Cesar Galindo) Can we view $\chi$-crossed braided categories as ``braided objects in $\chi$-crossed categories?''
  \item (Shawn Cui) A $G$-crossed braided category is a monoidal $2$-category with invertible objects.
    Does this generalize to $\chi$-crossed braided categories?
\end{itemize}

\section{H\'ector Mart\'in Pe\~na Pollastri -- The Relationship of Bicrossed Products of Fusion Categories and Extensions}

Slide talk.
We study algebraic objects by decomposing them into simpler objects and understanding how to build larger objects from these simple parts.

For groups, we have a few nice methods:
\begin{itemize}
  \item Extensions $H \to G \to K$.
  \item Exact factorizations $H \to G \leftarrow K$ with $G = H \cdot K$ and $H \cap K = \{ e \}$.
\end{itemize}
Both of these methods generalize to fusion categories.

\subsection{Exact sequences of fusion categories}

We may make sense of an exact sequence of fusion categories $\Ac \xrightarrow{i} \Bc \xrightarrow{F} \Cc$.
However, the na\"ive definition forces $\Ac$ to have a fiber functor.
Etingof--Gelaki proposed to instead look at sequences
\[
  \Ac \xrightarrow{i} \Bc \xrightarrow{F} \Cc \otimes \Endc(\Mc)\op
\]
and interpret ``$\ker F$'' as the preimage of $\Endc(\Mc)\op$.
We may make sense of equivalences of these and thus define $\Ext(\Ac'', \Ac', \Mc)$ as equivalence classes of extensions of $\Ac''$ by $\Ac'$ with respect to $\Mc$.

To prove theorems, we should consider special cases.
Let's think about abelian extensions.
Abelian extensions $\Rep G \to \Dc \to \Cc$ (with respect to $\Mc = \Vect$) are fully classified by crossed extensions.
We may write $\Dc = \Cc^{(G, \pi)}$.
This can be understood as a ``crossed product.''

\subsection{Exact factorizations}

Exact factorizations of groups come from bicrossed products of ``matched pairs'' $(H, K, H \leftarrow H \times K \to K)$.
We can give a similar description for fusion categories.
We can also decategorify this to give descriptions of the corresponding fusion rings.
It's not clear if the associators of the corresponding categories always arise via bicrossed products.

It's possible to make sense of equivavlence classes of exact factorizations.
One can then show that (when the module category $\Mc$ is $\Vect$ or $\Ac$) exact factorizations forrespond to exact sequences.
Under this equivalence, the crossed extensions above correspond to bicrossed products with $\Vect_G$.
In particular, every exact factorization $\Cc \cdot \Vect_G$ is equivalent to a bicrossed product.
One may work this out explicitly for Tambara--Yamagami categories.

\section{David Green -- Braidings for Non-Split Tambara--Yamagami Categories over the Reals}

This is based on joint work with Yoyo Jiang and Sean Sanford.

\begin{ex}
  The categories $\Rep_\CC(D_4)$ and $\Rep_\CC(Q_8)$ are Tambara--Yamagami categories.
\end{ex}

We'll want to understand the situation over $\RR$.
Here things are more complicated because the endomorphisms of a simple object need not be the ground field.
Some examples to consider are $\Bim_\RR(\CC)$ and $\Rep_\RR(\ZZ / 4 \ZZ)$.

One can classify Tambara--Yamagami categories over $\RR$ using a diagram the speaker drew.
(I took a photo.)

Consider a Tambara--Yamagami category $\TY_\CC(A, \chi, \tau)$ over $\CC$.
If this is braided, then:
\begin{itemize}
  \item We must have $A = \ZZ_2^n$.
  \item Braidings are classified by quadratic forms $\sigma$ on $A$ with $\delta \sigma = \chi$.
  \item The group of braided automorphisms is given by $\Aut(A, \sigma)$.
\end{itemize}

In the split real case, $\chi$ must be $\RR$-valued (because it is used to define associators).
The braiding gives rise to a number $\sigma_3(1) \in \RR$, and we need $\sigma_3(1)^2 = \tau \sum(\sigma)$ where $\sum(\sigma)$ denotes the Gauss sum.
This produces an extra condition for a braiding to exist: we need $\sgn(\sum(\sigma)) = \sgn(\tau)$.
The group of braided automorphisms is still given by $\Aut(A, \sigma)$.

Handling other cases proceeds along similar lines.

\section{Eric Samperton -- Complexity Dichotomies for TQFT Invariants}

Slide talk.

\subsection{Warmup}

Let $G$ be a finite group.
If $\Gamma$ is any finitely presented group, let $\#H(\Gamma, G)$ be the number of homomorphisms $\Gamma \to G$.

\begin{thm}
  The problem of counting $\#H(\Gamma, G)$ is in functional polynomial time if and only if $G$ is abelian.
  Otherwise the problem is $\#P$-complete.
\end{thm}

The goal of this talk is to prove a similar result for invariants of $3$-manifolds.
Here we replace $\Gamma$ by a (combinatorially presented) $3$-manifold and $G$ by a fusion category.

\subsection{Motivation from topological quantum computing}

Kitaev proposed a quantum error-correcting ``toric code.''
This has fault-tolerant ``anyon braiding'' operations (which relate to topological invariants of links).
More generally, for cellulated surfaces, we may define ``surface codes.''

The toric code is not powerful enough to build a universal quantum computer.
To get such a universal quantum computer, we need to work with non-abelian theories.
Anyon braiding in $U_q(\sl_2)$ is $BQP$-universal.
The $BQP$-universal quantity here is a certain non-topological normalization of the Jones polynomial.

We can also do a more powerful version of TQC using ``adaptive charge measurements.''
This allows us to fuse anyons, measure the charge (collapsing the wave function), and use this to control future braidings.
This is still fault-tolerant but no longer computes knot / $3$-manifold invariants.

There are two ways to implement TQC:
\begin{itemize}
  \item ``Software'' using quantum error-correcting codes.
    This is hardware agnostic, but encoding / decoding is difficult.
    Mathematically, this relates to Turaev--Viro theories.
  \item ``Hardware'' using new phases of topological quantum matter.
    This requires a firm control of the hardware, but encoding / decoding is easier.
    Mathematically, this relates to Reshetikhin--Turaev theories.
\end{itemize}

Recall that Turaev--Viro TQFTs are fully extended TQFTs associated with unitary fusion categories $\Cc$.
These compute invariants of closed oriented triangulated $3$-manifolds via state sums.
When $\Cc = \Rep(G)$, the invariant is $\#H(\pi_1(M), G) / \#G$.
Note that $3$-manifold groups $\pi_1(M)$ are ``class $2$ solvable'' so the counting problem is easier in this case.

Reshetikhin--Turaev TQFTs are partially extended TQFTs defined using unitary \emph{modular} fusion categories.
These compute invariants of closed oriented $3$-manifolds with \emph{surgery diagrams} via state sums.
For spherical fusion categories $\Cc$ the Reshetikhin--Turaev theory for $Z(\Cc)$ and Turaev--Viro theories for $\Cc$ agree.
However, note that Reshetikhin--Turaev theories are more general: not every unitary modular fusion category arises as $Z(\Cc)$.
We'll focus on Reshetikhin--Turaev theories.

\subsection{Anyon classification}

Here are five types of ``anyon classification'' problems:
\begin{enumerate}
  \item Classify unitary modular fusion categories (ideally ``up to finite group theory'')
  \item Classify simple objects in a unitary modular fusion category according to the image of the braid group actions (dense, finite, ...).
    The Naidu--Rowell ``Property F conjecture'' is relevant here.
  \item Classify modular fusion categories according to the computational complexity of exact computation of Reshetikhin--Turaev invariants.
    Given a fixed $\Bc$, how hard is it to compute Reshetikhin--Turaev invariants of surgery diagrams?
    We may also try to restrict to certain classes.
    The speaker showed that the problem is hard for hyperbolic manifolds whenever it is difficult in general.
  \item Classify modular fusion categories according to the computational complexity of \emph{approximate} computation of Reshetikhin--Turaev invariants.
    This requires us to choose a definition of ``approximate.''
    One may look at specific problems relevant to universal TQC.
  \item Classify unitary modular fusion categories according to whether they support universal TQC with adaptive charge measurements.
    This is the most important for applications!
\end{enumerate}

We'll focus on problem 3 here as a useful starting class for the (more important) problems 4 and 5.
A result of Kuperberg gives a key relationship between these problems.

\subsection{Preliminary results}

Let $\Bc$ be a modular fusion category.
One may define a computational problem $\#CSP(\Fc_\Bc)$ of ``exactly contracting tensors built from $\Bc$.''
This is either:
\begin{itemize}
  \item Solvable in polynomial time, or
  \item $\#P$-hard.
\end{itemize}
In the former case, Reshetikhin--Turaev invariants may be computed in polynomial time.
In the latter case it is unlikely that the tensor contraction problem may be solved efficiently using a quantum computer.
We don't know whether the Reshetikhin--Turaev invariants may be computed more efficiently.
It would also be interesting to develop results for probabilistic computations.

The proof relies on a difficult theorem of Cai--Chen related to weighted constraint satisfaction problems.
The idea is to turn the Reshetikhin--Turaev problem into the Cai--Chen problem.
This is easiest to do in the Turaev--Viro case.

Note: it is difficult to determine whether, for a given $\Bc$, the problem $\#CSP(\Fc_\Bc)$ is solvable in polynomial time!

\subsection{Some conjectures}

If we fix bounds on the first $\ZZ_2$ Betti number of $M$, we may compute Reshetikhin--Turaev invariants of $M$ for Tambara--Yamagami categories efficiently.
This leads us to ask: when can we turn exponential sums into sums that are easier to evaluate?

\begin{thm}[S]
  Fix a finite group $G$ which is nilpotent of class $2$.
  Let $M$ be a triangulated / surgery $3$-manifold.
  There is a polynomial time algorithm for counting $\#H(\pi_1(M), G)$.
\end{thm}

A rough conjecture is that exact computation of Reshetikhin--Turaev invariants for $\Bc$ is easy if and only if $\Bc$ is ``nilpotent of class at most $2$.''
This can be defined precisely in terms of gaugings of abelian anyons.
The speaker has an efficient algorithm when $\Bc$ is nilpotent of class at most $2$.

One would guess that $\Bc$ supports $BQP$-universal TQC with adaptive charge measurement if and only if the corresponding manifold invariants are $\#P$-hard.

\section{Eric Rowell -- Braids and Categories as Quantum Symmetries}

Slide talk.
There's a good dictionary between unitary modular categories and anyonic systems.
In particular, anyons are objects of unitary modular categories.

\subsection{Modular fusion categories}

We like to think about many examples of modular fusion categories:
\begin{itemize}
  \item The Fibonacci category.
  \item The Ising category.
  \item Pointed ccategories associated with $(A, q)$ where $A$ is an abelian group and $q$ is a quadratic form.
  \item Unitary modular categories arising from quantum groups at roots of unity.
\end{itemize}

As always, one has methods for producing new modular fusion categories from existing examples:
\begin{itemize}
  \item Gauging
  \item Condensation / localization
  \item Zesting / condensed fiber products
\end{itemize}

There is a mostly complete classification of unitary modular fusion categories up to rank $12$.
(There are some potential examples of classification data in ranks $11$ and $12$ where the speaker isn't sure whether examples exist.)

\subsection{The property F conjecture and some related questions}

\begin{conj}[Property F conjecture]
  Let $a$ be any anyon type.
  Then $|\rho_a(B_n)| < \infty$ if and only if $d_a^2 \in \ZZ$.
\end{conj}

This has been proved in many cacses (e.g.\ quantum groups, weakly group-theoretical cases).
Anyons satisfying the conclusion of the conjecture cannot be braiding universal.
Cui and collaborators suggested ``measurement assisted protocols'' that can fix this.

Let $a$ be an anyon corresponding to an object $V \in \Cc$.
We'd like to consider $R \in \End(V \otimes V)$ and understand the behavior of $\rho_a$ in terms of $R$.

\begin{conj}[Matrix version of property F conjecture]
  Let $R$ be a unitary Yang--Baxter operator.
  Then $\rho^R(B_n)$ is \emph{virtually abelian}, i.e.\ there exists a finite-index abelian subgroup of $\rho^R(B_n)$.
  In particular, $\{ R \}$ is never universal.
\end{conj}

One might try to solve this by classifying Yang--Baxter operators.
This is hard!
Hietarinta classified them in the case $\dim V = 2$, but nobody has managed to do this in higher dimensions.

One approach is to categorify the problem.
We may form a monoidal category of braids $\BB = \sqcup_n B(B_n)$.
There's also a monoidal category of matrices $\Mat$ (the skeleton of $\Vect$).
Thus a Yang--Baxter operator $R$ corresponds to a strict monoidal functor $\BB \to \Mat$ (up to non-monoidal equivalence).
This leads one to consider the category $\MonFun(\BB, \Mat)$.
More generally, we can consider $\YB(\Cc)$, the category of \emph{Yang--Baxter objects} of $\Cc$.

\begin{thm}
  If $\Cc$ is braided, then $\YB(\Cc)$ is monoidal with unit $(\onebb, \id_{\onebb \otimes \onebb})$.
\end{thm}

The monoidal structure on $\YB(\Cc)$ is the ``lashing product.''
Note that $\YB(\Cc)$ doesn't have an initial or terminal object, so there are many strange subtleties.

To classify Yang--Baxter objects, we are interested in looking at certain subcategories $\Cc \subset \Mat$, e.g.\
\begin{itemize}
  \item Permutation matrices.
  \item Unitary matrices.
  \item ``Charge conserving'' matrices.
    In this case there is a complete classification of Yang--Baxter objects in terms of ``bicolored composition tableaux.''
\end{itemize}

\section{Shawn Xingshan Cui -- On Some Problems in Topological Quantum Computing}

\subsection{Basics of TQC}

Recall that an anyon system corresponds to a unitary modular tensor category $\Cc$.
Here:
\begin{itemize}
  \item Types of anyons correspond to simple objects.
    In particular, the vacuum corresponds to the monoidal unit.
  \item Fusion rules correspond to decompositions of tensor products into sums of simple objects.
  \item The state space $V^{a_1 \dots a_n}_{c}$ corresponds to the Hom space $\Hom(c, a_1 \otimes \dots \otimes a_n)$.
  \item Splitting / fusion tree bases of state spaces correspond to bases of Hom spaces.
  \item $F$-symbols and $R$-symbols may be defined in terms of associators / braiding data.
\end{itemize}

Quantum computing has the following ingredients:
\begin{itemize}
  \item Qubits are $\CC^2$ (with orthonormal bases $\ket{0}, \ket{1}$).
    More generally we may consider qudits $\CC^d$.
    Quantum states are unit vectors in $(\CC^2)^{\otimes N}$.
  \item Quantum states are unitary transformations on $(\CC^2)^{\otimes N}$.
  \item Measurement is a probabilistic operation on states.
\end{itemize}

In TQC we take qubits to be state spaces $V^{a_1 \dots a_n}_c$.
Quantum gates are given by the images of braids under representations acting on $V^{a_1 \dots a_n}_c$.
TQC can compute the Jones polynomial, and in fact this is a universal problem.

\subsection{Universality}

As $N \to \infty$ we have $\dim V_b^{a^{\otimes N}} \to (d_a)^N$ for some appropriate $b$.
Here $d_a$ is the Frobenius--Perron dimension.
If $d_a = 1$, we say $a$ is an abelian anyon.
Otherwise $d_a$ is a non-abelian anyon.

Say that an anyon $a$ is braiding universal if, for some $b$, $\im(\rho)$ is dense in $\Urm(V_b^{a^{\otimes N}})$ for sufficiently large $N$.

\begin{thm}[Freedman--Larsen--Wang]
  For $k \neq 0, 1, 2, 4$, the anyon of type $1$ in $\SU(2)_k$ is braiding universal.
\end{thm}

\begin{thm}[A.\ Kaufmann--Cui]
  For $k \neq 0, 1, 2, 4$, the anyon of type $1$ in $\SU(2)_k$ on one qubit is universal through double braiding, i.e.\ the image is dense in $\Urm(2)$.
\end{thm}

Let's think about the remaining cases:
\begin{itemize}
  \item $k = 0$ is trivial.
  \item $k = 1$ is the semion theory, which is abelian (and therefore not useful).
  \item $k = 2$ is the Ising theory, which is expected to be efficiently simulatable using a classical computer.
  \item For $k = 4$, we look at qutrit models.
    This is universal when we also allow measurement of total type of two anyons.
    We call this ``resource-assisted universal.''
\end{itemize}

It would be interesting to classify the resource-assisted universal theories.

\begin{ex}
  Consider the dihedral group $D_n$.
  Irreps of the quantum double are classified by conjugacy classes in $D_n$ together with irreps of their centralizers.
  The speaker has a Mathematica package for computing $F$-symbols and $R$-symbols of these quantum doubles.
  One can show that the double of $D_3 = S_3$ gives a resource-assisted universal model of TQC.
  There are other results for $D_6$ and $D_5$.
\end{ex}

\subsection{Leakage-free braiding}

Note that $V_b^{a^{\otimes N}}$ does not naturally decompose as a tensor product.
Instead, we work with subspaces of $V_b^{a^{\otimes N}}$ that do decompose as tensor products.
Say a braid is \emph{leakage-free} if it preserves such a subspace and \emph{entangled} if it does not decompose as a tensor product of unitary maps followed by a swap.

\begin{qn}
  Does the Fibonacci model have an entangled leakage-free braiding gate?
\end{qn}

None have been found among braids with word length less than $10$.

\begin{conj}
  An anyon is braiding universal if and only if it does not support any leakage free braiding gates.
\end{conj}

\subsection{Quantum error correction}

Anyon systems often come from exactly solvable lattive models.

\begin{ex}
  The toric code arises in this way and gives abelian anyons.
  Ground states for the toric code are locally indistinguishable, so the toric code is error-correcting.
\end{ex}

\begin{ex}
  The Kitaev quantum double model also arises in this way.
  In this case, the Hamiltonian is ``frustration free,'' so the ground states should be locally indistinguishable.
  The conjecture ``frustration-free implies error-correcting'' has not been proven mathematically.
  The speaker has some results in this direction for the Kitaev quantum double model and twisted variants.
  These are enough to prove that the twisted Kitaev model is error-correcting.
  One may show that the twisted Kitaev model is a ``local topological order.''
\end{ex}

\section{Julia Plavnik -- The Condensed Fiber Product and Zesting}

This is based on joint work with Colleen Delaney, Cesar Galindo, Eric Rowell, and Qing Zhang.

\subsection{Motivation}

We want to classify modular fusion categories.
In particular, the current project came out of understanding modular fusion categories of Frobenius--Perron dimension $36$, where some interesting number-theoretic examples appear.
Zesting turns out to be useful for:
\begin{itemize}
  \item Categorifying fusion rings
  \item Realizing modular data
  \item Distinguishing ``modular isotopes'' (categories with the same modular data)
\end{itemize}

Zesting is also quite concrete: we may describe ranks, Frobenius--Perron dimensions, modular data, and many other properties of the resulting category.
However, we can't control the central charge or Witt class.

Finally, zesting may be used to understand Kitaev's 16-fold way for minimal modular extensions.
There are two orbits within this way, and the two are related by zesting.

\subsection{Setup}

Let $\Cc$ be a ribbon fusion category.
This comes with:
\begin{itemize}
  \item A fusion ring $K_0(\Cc)$.
  \item Associators $\alpha_{X,Y,Z}$ satisfying the pentagon equation.
  \item A braiding $c_{X,Y}$ satisfying hexagon equations.
  \item A twist $\theta_X: X \xrightarrow{\sim} X$ satisfying $\theta_{X \otimes Y} = (\theta_X \otimes \theta_Y) \circ c_{Y,X} \circ c_{X,Y}$.
\end{itemize}

\subsection{Zesting}

Suppose we have a grading $\Cc = \oplus_{g \in G} \Cc_g$ with $\otimes: \Cc_g \otimes \Cc_h \to \Cc_{gh}$.
Assume for simplicity that $\Cc$ is braided.

For zesting, we want to keep the category the same but adjust the tensor product on $\Cc$ so that
\[
  X_g \tilde{\otimes} Y_h = X_g \otimes Y_h \otimes \lambda(g, h) \in \Cc_{gh}
\]
where $\lambda(g, h)$ is invertible in $\Cc_e$.
To make this new tensor product associative:
\begin{itemize}
  \item $\lambda(g, h)$ must come with a half-braiding $c_{\lambda(g,h),-}$, i.e.\ $\lambda(g,h)$ must live in some ``relative center.''
    In our case this comes from the braiding on $\Cc$.
  \item $\lambda(g, h)$ must satisfy a cocycle condition: there should be isomorphisms 
    \[
      \nu(g, h, k): \lambda(g, h) \otimes \lambda(gh, k) \xrightarrow{\sim} \lambda(h, k) \otimes \lambda(g, hk).
    \]
  \item These $\nu$ maps must satisfy the pentagon equation.
    We end up seeing that $\nu$ must be a $3$-cochain, i.e.\ there is an obstruction in $H^4$.
  \item We also like to normalize everything so that the trivial component is unchanged.
\end{itemize}
We get a new fusion category $\tilde{\Cc} = \Cc^{(\lambda,\nu)}$ which is also an extension of $\Cc_e$.

We'd also like to adjust the braiding $c$.
This uses $t: \lambda(g, h) \to \lambda(h, g)$, though as above there may be obstructions.

If $\Cc$ is furthermore a ribbon fusion category, we want to adjust the twist $\theta$ so that $\tilde{\theta}_{X_g} = f(g) \theta_{X_g}$ for some $f: G \to k^\times$ (with vanishing obstructions).
We end up with a new ribbon fusion category.

\begin{ex}
  Associative zesting transforms $\Vect_G$ into $\Vect_G^\omega$.
\end{ex}

As zesting doesn't change the underlying category, the rank and Frobenius--Perron dimension are unchanged.
We also have formulas for the associators, modular data, M\"uger center, etc.
Zesting may be viewed as a particular case of extension theory.

\subsection{$G$-crossed zesting}

What happens if $\Cc$ is a $G$-crossed braided fusion category?
Here $\Cc$ is $G$-graded but also comes with a $G$-action $T_h(-) = (-)^h$ satisfying $T_h(\Cc_g) = \Cc_{h\inv g h}$.
The braiding is now $G$-crossed, i.e.\
\[
  c_{X_g,Y_h}: X_g \otimes Y_h \xrightarrow{\sim} Y_h \otimes (X_g)^h.
\]

To zest such a category, we use the same formula for $\tilde{\otimes}$.
However, now we adjust $\alpha$ using 
\[
  \nu: \lambda(g, h)^k \otimes \lambda(gh, k) \to \lambda(h, k) \otimes \lambda(g, hk).
\]

\begin{thm}
  The resulting $\Cc^{(\lambda,\nu)}$ is a fusion category.
\end{thm}

By a theorem of Davydov and Nikshych, central $G$-extensions give $G$-extensions of braided subcategories $\Bc \subset \Cc$ with lifts $\Bc \to \Zc(\Cc)$.
The upshot for us is that $G$-crossed extensions correspond to central $G$-extensions.

\begin{thm}
  The inclusion $\Cc_e \subset \Cc^{(\lambda,\nu)}$ is a central $G$-extension.
  Thus $\Cc^{(\lambda,\nu)}$ is a $G$-crossed braided fusion category and we may explicitly understand the $G$-action and braiding.
\end{thm}

\begin{thm}
  Let $\Bc$ be a braided fusion category.
  Then two $G$-crossed braided fusion categories have the same $G \to \Mod(\Bc)^\times$ if and only if they are related by $G$-crossed zesting.
\end{thm}

\subsection{The condensed fiber product}

Let $B$ be a finite abelian group.
Fix $z \in Z(B)$ with $z^2 = 1$,
We get a quadratic form $q_z$ on $\hat{B}$ by $q_z(\phi) = \phi(z)$.
Thus we obtain a symmetric pointed fusion category $\Bc_z = \Cc(\hat{B}, q_z)$.

We's like to consider a non-degenerate(ly $B$-graded) fusion category $\Cc$ with $\Bc_z \subset \Cc_e$.
We have a decomposition $\Cc = \oplus_{b \in B} \Cc_b$ where $\Cc_b = \bset{X}{c_{\phi,X} \circ c_{X,\phi} = \phi(b) \forall \phi \in \hat{B}}$.
Here we have $\Bc_z \subset \Cc_e \cap \Cc_e'$.

Suppose we also have a non-degenerate $\Dc$ with $\Bc_z \subset \Dc$.
We have $\Rep(\Bc) \equiv \nabla \Bc_z= \angles{(\phi, \phi\inv)} \subset \Cc \boxtimes \Dc$.
We define 
\[
  \Cc \star^B \Dc = (\Cc \boxtimes \Dc)^\loc_{\Fun(\Bc)} = (\oplus_{b \in B} \Cc_b \boxtimes \Dc_b)_B = (\Cc \boxtimes^B \Dc)_B.
\]
This is the \emph{condensed fiber product.}
By construction, $\Cc \star^B \Dc$ is non-degenerate.
Here $(\Bc_z \boxtimes \Bc_z)_B = \Bc_Z \subset (\Cc \star^B \Dc)_e \cap (\Cc \star^B \Dc)_e'$.

We may consider the category  $\Sc^\nd(B, \Bc_z)$ of non-degenerate $\Cc$ with $\Bc_z \subset \Cc$.
This category is monoidal with operation $\star^B$ and unit $\Zc(\Bc_z)$.

\begin{thm}
  If $\Pc \in \Sc^\nd(B, \Bc_z)$ is pointed and $\dim(\Pc) = |\Bc|^2$, then $\Pc$ is a zesting of $\Zc(\Bc_z)$.
\end{thm}

\begin{thm}
  If $\Cc \in \Sc^\nd(B, \Bc_z)$, then any $B$-zesting of $\Cc$ may be obtained as $\Cc \star^B \Pc$ with $\Pc \in \Zc(\Bc_z)^{(\lambda,\nu,t)}$.
\end{thm}

\section{Abigail Watkins -- From $G$-Equivariantization to the Category of Wilson Lines and Back Again}

Slide talk.
Based on joint work with Sebastian Heinrich, Julia Plavnik, and Ingo Runkel.

\subsection{Overview}

Suppose we are given a (faithfully $G$-graded) $G$-crossed ribbon category $\hat{\Bc}$.
We can equivariantize to get $\hat{\Bc}^G$, or we can use an ``orbifold datum'' $\AA \in \Bc = \hat{\Bc}_e$ to get a category of Wilson lines $\Bc_\AA$.
The claim is that $\hat{\Bc}^G \simeq \Bc_\AA$ as ribbon categories.

Recall that $\hat{\Bc}^G$ is the ribbon tensor category with objects 
\[
  (X \in \hat{\Bc}, (\eta_g^X: \phi(g) X \xrightarrow{\sim} X)\big)
\]
such that the equivariant structure maps $\eta_g^X$ form the expected commutative squares.
The tensor products, braidings, and twists are all defined naturally.

\subsection{Orbifold data}

Orbifold data $\AA \in \Cc$ allow us to modify ribbon categories via a ``Wilson lines construction.''
More precisely, $\AA$ consists of:
\begin{itemize}
  \item A $\Delta$-separable symmetric Frobenius algebra $A$,
  \item an $(A, A \otimes A)$-bimodule $T$, and
  \item several structure maps.
\end{itemize}

The category of Wilson lines $\Bc_\AA$ has objects given by $(A, A)$-bimodules $M$ together with several ``intertwining maps'' controlling $M \otimes_A T$.

If $\hat{\Bc}$ is a $G$-crossed ribbon map, we may produce an orbifold datum in $\Bc$.
These require some choices.

\begin{thm}
  With the above construction, there is an equivalence $\hat{\Bc}^G \simeq \Bc_\AA$.
  In particular the category $\Bc_\AA$ is independent of the choices made.
\end{thm}

\section{Marco Boggi -- A Generating Set for the Johnson Kernel}

\subsection{The Johnson group and the Torelli group}

Let $S_{g,m}$ be a surface of genus $g$ with $m$ punctures.
The \emph{mapping class group} of $S_{g,m}$ is
\[
  \Gamma(S_{g,m}) = \pi_0(\Diff^+(S_{g,m})) = \Diff^+(S_{g,m}) / \Diff^0(S_{g,m}).
\]
There is a natural surjection $\Gamma(S_{g,m}) \to \Sigma_m$ given by restricting to the punctures.
The kernel of this map is the \emph{pure} mapping class group $P\Gamma(S_{g,m})$.

There is also a natural surjection $P\Gamma(S_{g,m}) \to \Sp(H_1(\ol{S}, \ZZ))$.
The kernel of this map is the \emph{Torelli group} $I(S)$.

We'll focus on the case $g \geq 3$ and $m \leq 1$.
In this case we let $J(S) = \ker(P\Gamma(S) \to \Out(\pi_1(S) / \pi_1(S)^{T(S)}))$.
This $J(S)$ is the \emph{Johnson group}, and it contains $[I(S), I(S)]$ as a normal subgroup.
We may view $J(S)$ as generated by Dehn twists $\tau_\gamma$ along separating simple closed curves $\gamma$.

There is a short exact sequence
\[
  \begin{tikzcd}
    1 \rar & J(S) / [I(S), I(S)] \rar & I(S)^\ab \rar & I(S) / I(S) \rar & 1.
  \end{tikzcd}
\]
where:
\begin{itemize}
  \item $J(S) / [I(S), I(S)]$ is a finite $\ZZ/2$-torsion abelian group, and
  \item $I(S) / J(S)$ is a torsion-free finitely-generated abelian group.
\end{itemize}

For $g \geq 3$, we have $I(S) = \angles{\tau_{\gamma_1} \tau_{\gamma_2}\inv}^{\Gamma(S)}$ where $\gamma_1$ and $\gamma_2$ are taken to be neighboring ``cuffs'' in a pants decomposition.
Here $\angles{}^{\Gamma(S)}$ denotes the smallest normal subgroup containing these elements.
(There's a picture but I can't \TeX it up in time.)

One may also show that $J(S) = \angles{\tau_{\gamma_1} \tau_{\gamma_2}}^{\Gamma(S)}$ where $\gamma_i$ bounds a genus $i$ surface in $S$.

\subsection{Results}

Can we find a smaller generating set for the Johnson group?

\begin{thm}
  The Johnson group $J(S)$ is generated by Dehn twists about simple closed curves which bound:
  \begin{itemize}
    \item An unpunctured genus $1$ subsurface,
    \item A $1$-punctured genus $1$ subsurface,
    \item An unpunctured genus $2$ subsurface, or
    \item A disc bounding $2$ punctures.
  \end{itemize}
\end{thm}

Note that this is a true generating set, not a normal generating set!

The proof proceeds by induction and a modification of Johnson's original argument.
Let $A_{g,m}$ be the claim for $S_{g,m}$.

\begin{lem}
  For $g \geq 5$, we have $A_{g,1} \Rightarrow A_{g+1,0}$.
\end{lem}

\begin{lem}
  We also have $A_{g,m} \Rightarrow A_{g,m+1}$.
\end{lem}

Combining these and noting that the statement is trivial when $g$ is small, we obtain the theorem.
The proof of the first lemma is standard, while the proof of the second is more difficult.

\section{Problem Session}

Here are problems focused on the talks from today and yesterday:
\begin{itemize}
  \item (Eric Samperton) There are 16 modular Ising categories.
    ES expects Ising invariants are hard, while SC expects classical simulation of Ising models is feasible.
    What are the tradeoffs between complexity versus computational power?
  \item Can we prove that Ising TQC is efficiently classically simulatable?
  \item (Kyle Kawagoe / Shawn Cui) How can we relate the braiding approach to adaptive measurements depending on super-selection sectors to post-selection?
    How can we tell whether a given modular fusion category allows us to implement a ``controlled braiding'' based on adaptive measurements?
  \item (John Nolan) Are there interesting examples of zesting tensor categories when the grading is infinite?
  \item (Julia Plavnik) What is the precise relationship between zesting and condensed fiber products?
    Is there a counterexample?
  \item (Shawn Cui) Does the TQFT invariant complexity dichotomy work in arbitrary dimension?
    ES pointed out that TQFT invariants of surfaces are always computable in polynomial time (since surfaces can be triangulated to be long and thin).
    ES thinks the question in $4$d may be relatively quick to answer.
  \item (Kyle Kawagoe) Does there exist a circuit from Kitaev to Levin--Wen for Hopf algebras?
  \item (Alexei Davydov) Do there exist two $3$-manifolds $M_1$, $M_2$ such that all Turaev--Viro invariants are the same but not all Reshetikhin--Turaev invariants are the same?
    This somehow relates to Witt group homomorphisms.
\end{itemize}

\section{Siddharth Vadnerkar -- $G$-Crossed BTC of Symmetry Defects}

Slide talk.
Based on joint work with Kyle Kawagoe and Daniel Wallick.

\subsection{The problem}

\begin{conj}
  $G$-crossed BTCs describe low-energy physics of $G$-symmetric 2d quantum spin systems.
\end{conj}

The story in 1d was explained by M\"uger.
The proof in 2d is the subject of this talk.

\subsection{M\"uger's work}

M\"uger focused on 1d spin systems with $G$-action.
Consider an infinite real line with each point $i \in \ZZ$ labeled by $\Ac_i = \CC^n$.
On this we may build a ``quasi-local algebra'' $\Ac = \otimes_{\textrm{intervals } I} \otimes_{i \in I} \Ac_i$.
We assume there is a gapped ground state $\omega_0: \Ac \to \Cc$.
Let $\beta^i_g$ describe a $G$-action on site $i$, and assume the ground state is $G$-invariant, i.e.\ $\omega_0 \circ \beta_g = \omega_0$.

Given a site $i$, we may describe a $g$-symmetry defect $\beta_g^{R_i}$ where we have $\beta_g$ act only on sites to the right of $i$.
The operator $\beta_g^{R_i}$ is $g$-localized in some $I \ni i$.
We may send $\beta_g^{R_i}$ to $\beta_g^{R_j}$ via some local unitary.
Thus we get $\omega_0 \circ \beta_g^{R_i} \circ \beta_g^{R_j} \approx \omega_0 \circ \beta_{gh}^{R_k}$, i.e.\ we may pass between these states using local operators.

More generally, we say that $\Pi \in \End(\Ac)$ is a $g$-defect if it is $g$-localized (in some interval) and transportable (terms both defined in the talk).
If $\pi_g$ is a $g$-defect and $\rho_h$ is an $h$-defect, then $\pi_g \circ \rho_h$ is a $gh$-defect.
This definition of $g$-defect turns out to work well!

Let $\Delta^g$ be the set of all $g$-defects, and say $\Hom(\pi_g, \rho_g) = \bset{T \in \Ac}{T \pi_g(-) = \rho_g(-) T}$.
We define a category $\Delta^G = \oplus_{g \in G} \Delta^g$.
This has a monoidal structure given by composition in $\End(\Ac)$.

\begin{thm}[M\"uger]
  The category $\Delta^G$ is a $G$-crossed braided tensor category.
\end{thm}

\begin{rmk}
  One can really think of this as a sketch of what's happening on the boundary.
\end{rmk}

\subsection{What happens in 2d}

Everything is different now!
Defects include domain walls and line operators.
Furthermore, there are multiple ways for defects to go off to infinity.

To simplify things, let's fix the behavior of our line defects at infinity.
We also consider ``allowed cones'' $\Lambda$ such that some translate of the line defect doesn't intersect the cone.
Given an allowed cone $\Lambda$, we can still make sense of left halves $L_\Lambda$ and $R_\Lambda$.
We imagine that there is a hypothetical ``spin chain at infinity,'' and the allowed cones describe intervals in this spin chain.
It's easiest to see with a picture but I can't TikZ it up.

Say that $\pi \in \End(\Ac)$ is a $g$-defect if it is $g$-localized (in some allowed cone) and transportable.
As before, the composition of $g$-defects and $h$-defects is a $gh$-defect.
We can imitate M\"uger's definition of a category, though there are many subtleties!

\begin{thm}
  We obtain a $G$-crossed braided $C^*$-tensor category $\Delta^G$.
\end{thm}

The special case of $\Delta^1$ gives a BTC of anyons.
This agrees with many known constructions from the literature.
We can also understand fractionalization classes $\eta$ from $\Delta^G$.
The class $\eta$ is stable under local perturbations.

\section{John S.\ Nolan -- Geometric Construction of Quiver Tensor Products}

This was my own talk and I only used paper notes, sorry!

\section{Maxine Calle -- Cut and Paste Theory of Manifolds and Cobordism}

Slide talk.
This is joint work with Maru Sarazola.

\subsection{SK-manifolds}

We want to study manifolds up to ``cut and paste operations,'' described by \emph{SK moves}.
``Hilbert's third problem for manifolds'' asks what algebraic measurements classify SK equivalence.
It turns out that all we need are the Euler characteristic and the signature (if we care about orientations).
More precisely, let $SK_d$ be the Grothendieck group of $d$-manifolds mod SK equivalence.

\begin{thm}[Karras--Kreck--Neumann--Ossa, 1970s]
  \[
    \SK_d = \begin{cases}
      0 & d \textrm{odd} \\
      \ZZ & d \equiv 2 \pmod 4 \\
      \ZZ \oplus \ZZ & d \equiv 0 \pmod 4.
    \end{cases}
  \]
\end{thm}

\subsection{SK-moduli spaces}

A more recent viewpoint is to study the moduli space of SK-equivalences and its homotopy type.
The SK-automorphisms of a manifold are particularly interesting.
One may study this using ideas from Zakharevich's ``scissors congruence $K$-theory.''

\begin{thm}[Hoekzema--Merling--Murray--Rovi--Semikina]
  There is a category $\Mfld_d^\partial$ with squares $K$-theory space $K^\square(\Mfld_d^\partial)$ satisfies
  \[
    K_0^\square(\Mfld_d^\partial) \cong \SK_d^\partial.
  \]
\end{thm}

We don't know what happens for the higher $K$-theory!
Furthermore, squares $K$-theory is hard to work with.

\begin{thm}[C--Sarazola]
  There is another Waldhausen-ish $K$-theory space $K(\Mfld_d^\partial)$ such that:
  \begin{itemize}
    \item There is a map $K(\Mfld_d^\partial) \to K^\square(\Mfld_d^\partial)$ which is an isomorphism on $\pi_0$.
    \item $K^1(\Mfld_d^\partial)$ controls SK-automorphisms.
    \item $K(\Mfld_d^\partial) \simeq \Omega B\Cob_d^f$.
  \end{itemize}
\end{thm}

Here $\Cob_d^f$ is the $\infty$-category with:
\begin{itemize}
  \item objects: closed $(d-1)$-manifolds
  \item morphisms: cobordisms with free boundary
  \item higher morphisms: diffeomorphisms relative to boundary
\end{itemize}
We have $\pi_0(B\Cob_d^f) = \Omega_d^f= 0$.

\begin{prop}[C--Sarazola, Merling--Raptis--Semikina]
  \[
    \pi_1(B\Cob_d^f) \cong \SK_d^\partial.
  \]
\end{prop}

The higher homotopy groups of $B\Cob_d^f$ remain unknown.

\section{Alice Kimie Miwa Libardi -- The Borsuk-Ulam Theorem for Filtered Spaces}

Slide talk.
Joint work with C.\ Biasi, D.\ de Mattos, E.\ dos Santos, and S.\ Ura.

\subsection{Background}

Let $S^m$ be the $m$-dimensional sphere and $A: S^m \to S^m$ the antipode map.

\begin{thm}[Borsuk-Ulam]
  If $f: S^m \to \RR^n$ is any continuous map, there exists $x \in S^m$ such that $f(x) = f(A(x))$.
\end{thm}

We'll focus on topological spaces $X$ with free $\ZZ_2$-action (corresponding to involutions $T: X \to X$).
We'd like to generalize Borsuk-Ulam by considering maps $f: X^n \to X^n$ and asking when there exists $x$ such that $f(x) = f(T(x))$.

\subsection{Results}

Consider the natural map $X/\ZZ_2 \to B\ZZ_2$.
Pulling back the generator $\alpha \in H^1(B\ZZ_2, \ZZ_2)$ to $X/\ZZ_2$ gives an Euler class $e_X \in H^1(X/\ZZ_2, \ZZ_2)$.
The speaker talked about how (non-)vanishing of Euler classes can be used to give obstructions to existence of equivariant maps.

The main theorem discusses a criterion for non-existence of equivariant maps.
This specializes to give a criterion in terms of filtrations of the spaces.
At around this point I stopped following things, sorry!

\section{Calvin McPhail Snyder -- What is the Target of a $2$-Functor out of \texorpdfstring{$\Tang(G, X)$}{Tang(G, X)}?}

Slide talk.

\subsection{Setup}

Let $\Tang$ be the category of oriented framed tangles.
Suppose $\Cc$ is a ribbon category, and let $\Tang_\Cc$ be the category of tangles coloredby objects of $\Cc$.
The \emph{Reshetikhin--Turaev construction} gives a functor $\Tang_\Cc \to \Cc$.

We can upgrade this all to a $G$-equivariant story.
Here $\Tang(G)$ is the category of oriented framed tangles with representations $\pi_1 \to G$.
More precisely, $\Tang(G)$ has objects generated by $(g, \pm)$ for $g \in G$, and the representations $\pi_1 \to G$ appear in the morphisms.
This is a $G$-ribbon category, i.e.\ a $G$-crossed braided category with a suitable twist functor.
The braiding here acts on one strand via conjugation.
For a $G$-ribbon category $\Cc$, the category $\Tang_\Cc(G)$ has tangles colored by objects of $\Cc$ respecting degrees.
Turaev (and others?) construct $\Tang_\Cc(G) \to \Cc$ giving invariants of tangles with $G$-structures.

\subsection{$\Tang(G, X)$}

We'll focus on $\Tang(G, X)$ where $X$ is a $G$-module.
This is a 2-category, not just a monoidal category.
Interesting examples include:
\begin{itemize}
  \item Quantizations of $\sl_2(\CC)$ Chern--Simons theory.
  \item Zesting.
\end{itemize}

Note that a monoidal category is a $2$-category with one object.
We may consider a graphical calculus for $2$-categories where:
\begin{itemize}
  \item Objects correspond to regions,
  \item $1$-morphisms correspond to points / strings, and
  \item $2$-morphisms correspond to boxes.
\end{itemize}

Let $\Tang(G, X)$ be the $2$-category with:
\begin{itemize}
  \item Objects are $x \in X$
  \item $1$-morphisms are $g: x \to xg$
  \item $2$-morphisms are tangles with labeling by $G$ and $X$, i.e.\ tangles plus representations of $\pi_1$ of the complement plus gauge data $x \in X$.
\end{itemize}

We may interpret this as a monoidal category with ``partially defined tensor product'' (we may only combine regions when we match the labelings for domains).

\subsection{Chern--Simons theory}

Consider Chern--Simons theory for $\SL_2(\CC)$.
More precisely, let $M$ be a closed $3$-manifold and consider a representation $\rho: \pi_1(M) \to \SL_2(\CC)$ (corresponding to a hyperbolic structure on $M$).
Fix a flat $\sl_2$-connection on $M$ with holonomy $\rho$.
We may define a Chern--Simons functional $S(A)$, where $S(A)$ is independent of the choice of $A$ modulo $4 \pi^2 i$.
To actually compute this, we:
\begin{itemize}
\item Triangulate $M$.
\item Use $\rho$ to find geometric shapes of tetrahedra.
\item Add dilogarithms on each tetrahedron.
\end{itemize}

If $M$ has a boundary (e.g.\ $M = S^3 \setminus \nu(K)$) we can still make sense of things, though we need to label the boundary by defects?
Actually doing this can be nontrivial.

\begin{thm}[MS, building off Inoue--Kabaya]
  One can (generically) define a $2$-functor $\Ic: \Tang(\SL_2(\CC), \CC^2) \to \Vect$ capturing the exponentiated Chern--Simons invariant $e^{S/2\pi}$.
\end{thm}

To make this work, one need to use region colors.

\begin{thm}[MS--Reshetikhin]
  There is a quantum invariant $\Zc_N(T, \rho)$ quantizing $\Ic(T, \rho)$.
  When $\rho$ is trivial this recovers Kashaev's invariant.
\end{thm}

This should relate to $\SL_2(\CC)$ quantum Chern--Simons and the volume conjecture.

\subsection{Zesting}

Let $\Cc$ be a modular tensor category and let $A$ be its universal grading group (the maximal $A$ such that $\Cc$ is faithfully $A$-graded).
Given zesting data $\zeta = (\lambda, \nu, t, \epsilon)$, we may define a new modular tensor category $\Cc^\zeta$.
This has the same objects as $\Cc$, but the tensor product and braiding are modified.

\begin{qn}
  How do the Reshetikhin--Turaev invariants constructed from $\Cc^\zeta$ relate to those constructed from $\Cc$?
\end{qn}

\begin{thm}[Delaney--Kim--Plavnik]
  If the coloring is $A$-homogeneous and $T$ is a link, then the RT invariants agree up to scalars.
\end{thm}

\begin{thm}[Delaney--MS]
  This may be generalized to tangles if we incorporate an extra grading.
  More precisely, instead of working with $\Tang(A)$, we need to use $\Tang(A, A)$.
\end{thm}

The upshot is that the associators in $\Cc^\zeta$ force us to keep track of $A$-gradings, and $\Tang(A, A)$ allows us to do this.
The ``scalar factors'' are accounted for by a functor $J_\zeta: \Tang(A, A) \to \Inv(\Cc_e)$ which may be computed in polynomial time.

\begin{qn}
  \begin{itemize}
    \item How does $J_\zeta$ relate to condensed fiber products?
    \item Is the image of $J_\zeta$ a $2$-category in a useful way?
    \item What are the right definitions / axioms for $(G, X)$-graded $2$-categories?
    \item How does this relate to the invariants of S\"ozer--Virelizier for homotopy $2$-types?
  \end{itemize}
\end{qn}

\section{David Chan -- Equivariant Categories and Cohomology Theories}

Slide talk.
Joint with Maxine Calle and Maximilien P\'eroux.

Goals for this talk:
\begin{itemize}
  \item Explain equivariant cohomology theories.
  \item Discuss $G$-symmetric monoidal categories.
  \item Explain how to use these to construct cohomology theories.
\end{itemize}

Let $G$ be finite throughout.

\subsection{Equivariant cohomology}

In the equivariant world, there are different kinds of ``holes'' for each dimension.

\begin{ex}
  For $G = \ZZ_2$, there are four different involutions on $S^2$.
  We want to understand these cohomologically.
  As $G$-modules, we have $H_2(S^2_\mathrm{triv}) = H_2(S^2_\mathrm{rot})$, so usual homology can't tell these apart.
\end{ex}

We may fix these by introducing additional grading.
More precisely, let $V$ be a real $G$-representation and let $S_V$ be its one-point compactification.
We'd like to grade things by $V \in \Rep(G)$ so that $H_V(S^V) = \ZZ$.

\subsection{Constructing equivariant cohomology theories}

Recall that Brown representability gives an equivalence between infinite loop spaces and (connective) cohomology theories.
Thus, to give a connective cohomology theory, it suffices to give a space and a sequence of deloopings.

\begin{ex}
  Let $G$ be an abelian group.
  Then $G \simeq \Omega^n B^n G$, and the corresponding cohomology theory is singular cohomology with coefficients in $G$.
\end{ex}

\begin{ex}
  The sphere spectrum $\SS = \colim_{n \to \infty} \Omega^n S^n$ represents framed bordism.
\end{ex}

This can be upgraded to the $G$-equivariant setting.
More precisely, if $X$ is a $G$-space, we define $\Omega^V X = \Map_*(S^V, X)$.
An infinite loop $G$-space is $X$ such that, for all $V$, there exist $X_V$ with $X \simeq \Omega^V X_V$ for all $V \in \Rep(G)$.
With this definition, there is a correspondence between infinite loop $G$-spaces and $G$-equivariant cohomology theories.

\subsection{Categories and cohomology theories}

Segal showed that every symmetric monoidal category $\Cc$ determines an infinite loop space $K(\Cc)$.
Thomason showed that every infinite loop space is equivalent to $K(\Cc)$ for some $\Cc$.
We'd like to do this in the $G$-equivariant setting, i.e.\ build infinite loop $G$-spaces out of equivariant SMCs.

The na\"ive approach is to use SMCs with $G$-action, i.e.\ functors from $BG$ to $\SMCat$.
This doesn't really work because the deloopings $K(\Cc) = \Omega^V X_V$ are hard to compute.

\begin{dfn}[Hill--Hopkins]
  A $G$-symmetric monoidal category $\Cc$ is:
  \begin{itemize}
    \item For all $H \subset G$, a SMC $\Cc_H$
    \item For $H \subset K \subset G$, induction and restriction functors between $\Cc_H$ and $\Cc_K$.
    \item For $K \subset G$ and $g \in G$, strong monoidal functors $\Cc_K \to \Cc_{gKg\inv}$.
    \item Higher coherence data controlling Mackey's formula, etc.
  \end{itemize}
\end{dfn}

\begin{ex}
  We may take $\Cc_H = \Rep(H)$ or $\Fin^H$.
\end{ex}

\begin{thm}[Calle--C--P\'eroux]
  Let $\Cc$ be a $G$-SMC.
  Then there exists an infinite loop $G$-space $K_G(\Cc)$ with underlying space $K(\Cc_1)$.
  Furthermore, every infinite loop space if $K_G(\Cc)$ for some $\Cc$.
\end{thm}

\begin{ex}
  $K_G(\Fin_*, \wedge)$ is $G$-framed bordism.
\end{ex}

\subsection{Where do $G$-SMCs come from?}

There's a category $\Span$ such that SMCs are the same as product preserving pseudofunctors $F: \Span \to \Cat$.
Here $\Span$ has:
\begin{itemize}
  \item Objects: finite sets
  \item Morphisms: spans of finite sets
  \item Products: sums of finite $G$-sets.
\end{itemize}

To define $G$-SMCs, we may imitate this but replace $\Fin$ by finite $G$-sets.
Then a $G$-SMC is a product-preserving pseudofunctor $\Span^G \to \Cat$.
These are ``Mackey functors in categories.''

\begin{qn}
  What is a $G$-braided monoidal category?
\end{qn}

\section{Kyle Kawagoe -- Levin--Wen as a Gauge Theory}

Joint with C.\ Jones, S.\ Sanford, D.\ Green, and D.\ Penneys.

Goals for the talk:
\begin{itemize}
  \item What are the hallmarks of a gauge theory?
  \item What is gauging?
  \item How can we write the Levin--Wen model as a gauged theory?
\end{itemize}

\subsection{Gauge theories}

\begin{ex}
  Let's start by recalling some basic electromagnetism.
  This has a few noteworthy properties:
  \begin{itemize}
  \item Here $\vec{E} = \nabla V$, i.e.\ $\vec{E}$ can be understood via differences in a potential.
  \item Furthermore, electromagnetism is quantized: electric flux through a closed surface is some multiple of the electron charge $e$.
  \item The gauge symmetry acts locally.
  \end{itemize}
\end{ex}

In general, a gauge theory will have a symmetry group $G$.
Charges are given by representations of $G$.
Fluxes are ``local violations of the theory'' / ``interesting boundary conditions on punctures.''

\begin{ex}
  The toric code is a $\ZZ_2$ gauge theory.
  We consider a grid where the edges are labeled by Hilbert spaces $\Hc_e = \CC^2$.
  The total Hilbert spaces is $\Hc = \otimes_e \Hc_e$.
  The Hamiltonian may be written
  \[
    H = -\sum_v A_v - \sum_p B_p
  \]
  where:
  \begin{itemize}
    \item At a vertex $v$, we have ``fluxes'' $A_v = Z_1 Z_2 Z_3 Z_4$ where
      \[
        Z = \begin{bmatrix}
          1 & 0 \\
          0 & -1
        \end{bmatrix}
      \]
      acts on each neighboring edge.
    \item At squares / plaquettes, we have ``charges'' $B_p = X_1 X_2 X_3 X_4$ where
      \[
        X = \begin{bmatrix}
          0 & 1 \\
          1 & 0
        \end{bmatrix}
      \]
      acts on each neighboring edge.
  \end{itemize}
  This can be understood as a lattice presentation of the double of $\ZZ_2$.
  Kramers--Wannier gauging gives an relationship between this and the ``trivial paramagnet.''
  More precisely, the $\ZZ_2$-invariants of the Hilbert space of the paramagnet correspond to the flux-free part of the Hilbert space of the toric code.
  There's also a relationship between the Hamiltonians.
  The upshot is that Kramers--Wannier turns a theory of domains into a theory of domain walls.
\end{ex}

\subsection{Levin--Wen}

Let $\Cc$ be a unitary fusion category.
The \emph{tube algebra} of $\Cc$ is 
\[
  \Tube(\Cc) = \bigoplus_{a,b,c \in \Irr(\Cc)} \Hom_\Cc(a \otimes b, c \otimes a).
\]
The algebra structure is given by ``stacking.''

The \emph{Levin--Wen model} has Hilbert space
\[
  \Hc = \otimes_v \Hc_v \textrm{ where } \Hc_v = \oplus_{a,b,c,d \in \Irr(\Cc)} \Hom_\Cc(a \otimes b, c \otimes d).
\]
The Hamiltonian is
\[
  H= -\sum_e A_e - \sum_b B_p
\]
I couldn't really follow the definition of $A_e$ and $B_p$ here.
One may relate this to a theory coming from the tube algebra via a ``Kramers--Wannier gauging'' operation.

\section{Shana Li -- Multivariable Knot Polynomials, the \texorpdfstring{$V_n$}{V\_n} Polynomials, and Their Patterns}

Slide talk.
Joint with Garoufalidis.

\subsection{Background}

A \emph{rigid $R$-matrix} is $R \in \Aut(V \otimes V)$ satisfying the Yang--Baxter equation and such that the partial transposes
\[
  \tilde{R^{\pm 1}} = (\epsilon \otimes \id \otimes \id) \circ (\id \otimes R^{\pm 1} \otimes \id) \circ (\id \otimes \id \otimes \eta)
\]
are invertible in $\End(V \otimes V)$.

\begin{thm}[Kashaev]
  Let $R$ be a rigid $R$-matrix.
  Then the corresponding Reshetikhin--Turaev functor gives an $\End(V)$-valued invariant of knots.
\end{thm}

The rigid condition is needed to deal with the case of caps / cups going in the ``wrong direction.''

\begin{thm}[Garoufalidis--Kashaev]
  Given a braided Hopf algebra with automorphisms, one may construct a rigid $R$-matrix.
\end{thm}

Nichols algebras are a useful class of braided Hopf algebras.
A \emph{Nichols algebra} is an algebra of the form $T(V) / J$ where $J$ is a maximal ideal not meeting $k \oplus V$.
Note that:
\begin{itemize}
  \item Nichols algebras of rank $1$ give colored Jones polynomials and ADO polynomials.
  \item Nichols algebras of rank $2$ give Links--Gould polynomials and $V_n$ polynomials.
\end{itemize}

For $V_n$ polynomials, the endomorphism of $V$ obtained via Kashaev's approach is a scalar multiple of $\id_V$.
The $V_n$ polynomial is given by the scalar.
In general, $V_n$ polynomials can get quite large.

\subsection{Results and conjectures}

The speaker computed $V_2$ polynomials for all knots with $\leq 16$ crossings.
Some steps were needed to optimize the computation:
\begin{itemize}
  \item Since the $R$-matrix is sparse, a ``divide and conquer method'' may be used.
  \item A certain tensor contraction path may simplify things.
\end{itemize}

The $V_n$ polynomials have many interesting patterns: 
\begin{itemize}
  \item There's a symmetry property.
  \item The $V_n$ polynomials conjecturally specialize to the Alexander polynomial.
  \item The $V_n$ polynomials conjecturally satisfy a bound given in terms of the Seifert genus: $\deg_t V_{K,n}(t, q) \leq 4g(K)$.
\end{itemize}

\begin{thm}[Garoufalidis--L]
  The genus bound inequality is an equality for $V_2$-polynomials with $\leq 16$ crossings.
\end{thm}

The speaker conjectures that $V_2$-polynomials detect the genus.

It is rather rare that two knots have equal $V_2$ polynomials.
In the cases checked, these have isomorphic knot Floer homology and isomorphic Khovanov homology.
These may all be Conway mutant knots to each other (so are generally hard to tell apart).
The $V_2$ polynomials do tell apart a large number of Conway mutant knots.
Most of these knots are also knot Floer homology thin and Khovanov thin, so being mutant implies that the corresponding invariants agree.

When $K$ is an alternating knot with $\leq 16$ coefficients, the polynomials $V_1(t, -q)$ and $V_2(t, -q)$ have nonnegative integer coefficients.
Is there a categorification of these?

\end{document}
